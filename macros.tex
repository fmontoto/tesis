% transaction {label}{transaction name}{list of inputs}
\makeatletter
\newcommand{\transaction}[0]{%
    \@transactionbegin
}
\newcommand\@transactionbegin[3]{%
    \begin{center}
        \label{#1}
        \begin{tabular}{| l | c |}
            \hline
            \multicolumn{2}{|c|}{\textbf{$T_{#2}$}  (\textbf{in:} \@transactionize{#3})} \\
            \hline
            \@transactioninputsit
}
\newcommand{\@transactioninputsit}[1]{%
    \ifx\stopinputs#1%
        \expandafter\@firstoftwo
    \else
        \expandafter\@secondoftwo
    \fi
    {\@transactioninputsend}
    {\@transactioninputsscript{#1}}%
}
\newcommand{\@transactioninputsscript}[2]{%
    $T_{#1}$ \textbf{in-script:} & #2 \\
    \hline
    \@transactioninputsit%
}
\newcommand{\@transactioninputsend}[0]{%
    \@transactionoutputsit
}
\newcommand{\@transactionoutputsit}[1]{%
    \ifx\stopoutputs#1%
        \expandafter\@firstoftwo
    \else
        \expandafter\@secondoftwo
    \fi
    {\@transactionoutputsend}
    {\@transactionoutputsscript{#1}}%
}
\newcommand{\@transactionoutputsscript}[2]{%
    \textbf{out-script} (#1): & #2 \\
    \hline
    \@transactionoutputsit %
}
\newcommand{\@transactionoutputsend}[0]{
        \end{tabular}
    \end{center}
}
% Take a comma separated list "a,b,c" and transform it into "T_{a},T_{b},T_{c}"
\newcommand{\@transactionize}[1]{%
    \@tempswafalse
    \@for\next:=#1\do
    {\if@tempswa\else\@tempswatrue\fi{$T_{\next}$}}
}
\def\stopinputs{}
\def\stopoutputs{}
\makeatother

% \begin{center}
    % \label{tx:presentation}
    % \begin{tabular}{| l | c |}
        % \hline
            % \multicolumn{2}{|c|}{\textbf{T_x}  (\textbf{in:} T_{y1} T_{y2})} \\
        % \hline
            % T_{y1} \textbf{in-script:} & \sigma_1 \\
        % \hline
            % T_{y2} \textbf{in-script:} & \sigma_2 \\
        % \hline
        % asdksd & asd \\
        % \hline
    % \end{tabular}
% \end{center}
