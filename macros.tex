% transaction {label}{transaction name}{list of inputs}
\makeatletter
\newcommand{\transaction}[0]{%
    \@transactionbegin
}
\newcommand\@transactionbegin[2]{%
    \begin{center}
        \begin{tabular}{| l | c |}
            \hline
            \multicolumn{2}{|c|}{\textbf{$T_{#1}$}  (\textbf{in:} \transactionize{#2})} \\
            \hline
            \multicolumn{2}{|c|}{\textit{Inputs:}} \\
            \hline
            \@transactioninputsit
}
\newcommand{\@transactioninputsit}[1]{%
    \ifx\stopinputs#1%
        \expandafter\@firstoftwo
    \else
        \expandafter\@secondoftwo
    \fi
    {\@transactioninputsend}
    {\@transactioninputsscript{#1}}%
}
\newcommand{\@transactioninputsscript}[3]{%
    \ensuremath{T_{#1}}[#2] & #3 \\
    \hline
    \@transactioninputsit%
}
\newcommand{\@transactioninputsend}[0]{%
    \multicolumn{2}{|c|}{\textit{Outputs:}} \\ \hline
    \@transactionoutputsit
}
\newcommand{\@transactionoutputsit}[1]{%
    \ifx\stopoutputs#1%
        \expandafter\@firstoftwo
    \else
        \expandafter\@secondoftwo
    \fi
    {\@transactionoutputsend}
    {\@transactionoutputsscript{#1}}%
}
\newcommand{\@transactionoutputsscript}[2]{%
    \makecell{(#1)} & \makecell{#2} \\
    \hline
    \@transactionoutputsit %
}
\newcommand{\@transactionoutputsend}[2]{
        \end{tabular}
        \captionof{table}{#2}
        \label{#1}
    \end{center}
}
% Take a comma separated list "a,b,c" and transform it into "T_{a},T_{b},T_{c}"
\newcommand{\@transactionize}[1]{%
    \@tempswafalse
    \@for\next:=#1\do
    {\if@tempswa\else\@tempswatrue\fi{$T_{\next}$}}
}
\newcommand{\transactionize}[1]{%
    \foreach \next [count=\ni] in {#1} {%
        \ifnum\ni=1%
            $T_{\next}$%
        \else%
            ,$T_{\next}$%
        \fi%
    }%
}
\newcommand{\fee}[1]{%
    \ensuremath{\mathcal{F}_{#1}}%
}
\newcommand{\signature}[1]{%
    \ensuremath{\mathcal{S}_{#1}}%
}
\newcommand{\timeout}[1]{%
    \ensuremath{\uptau_{#1}}%
}
\newcommand{\extra}[1]{%
    ... \guilsinglleft #1\guilsinglright\ ...
}
\def\stopinputs{}
\def\stopoutputs{}
\newcommand{\@calctotalbytes}[3]{%
    \newcounter{constCost} \setcounter{constCost}{#1}%
    \newcounter{perOracleCost} \setcounter{perOracleCost}{#2}%
    \newcounter{numOracles} \setcounter{numOracles}{#3}%
    \setcounter{totalbytes}{%
            \numexpr\theconstCost+(\thenumOracles*\theperOracleCost)\relax}%
}
\newcommand{\@calctotalcost}[2]{%
    \newcounter{bytesize} \setcounter{bytesize}{#1}%
    \newcounter{feeVal} \setcounter{feeVal}{#2}%
    \setcounter{totalCost}{%
            \numexpr\thetotalbytes*\thefeeVal\relax}%

}
\newcommand\calculatecosts[4]{%
    \newcounter{totalbytes} \@calctotalbytes{#1}{#2}{#3}%
    \newcounter{totalCost} \@calctotalcost{\thetotalbytes}{#4}%
    \theconstCost & \theperOracleCost & \thetotalbytes & \num{\thetotalCost}%
}
\newcommand\totalcostimpl[4]{%
    \newcounter{totalbytes} \@calctotalbytes{#1}{#2}{#3}%
    \newcounter{totalCost} \@calctotalcost{\thetotalbytes}{#4}%
    \num{\thetotalCost}%
}
\newcommand\satoshitousd[2]{%
    \newcounter{numsatoshis} \setcounter{numsatoshis}{#1}
    \newcounter{bitcoinusd} \setcounter{bicoinusd}{#2}
    \newcounter{result} \setcounter{result}{%
      \numexpr(\thenumsatoshis * \thebitcoinusd) / 100000000\relax}%
    \theresult %
}
\makeatother
