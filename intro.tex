\begin{intro}
Usualmente cuando hablamos de dinero imaginamos un billete o una moneda. Sin
  embargo, cuando decimos dinero digital es m\'as complejo hacernos una imagen.
  Generalmente lo asociamos a pagos electr\'onicos, en los cuales no vemos el
  billete o moneda. El dinero digital exhibe propiedades similares al f\'isico,
  se diferencia en que permite transacciones instant\'aneas e intercambios
  entre distintos pa\'ises sin importar fronteras.
  
Una criptomoneda es un tipo de moneda digital, donde se utilizan medios
  criptogr\'aficos para asegurar las transacciones y para controlar la
  creaci\'on de nuevas unidades. La idea de utilizar herramientas
  criptogr\'aficas en monedas digitales surge como t\'opico de
  investigaci\'on en los a\~nos 80, cuando David Chaum \cite{chaum1983blind}
  introduce una nueva primitiva para realizar pagos imposibles de rastrear.
  
En el a\~no 2012 el banco central europeo \cite{bcentraleuro} define una moneda
  virtual como un tipo de moneda digital desregulada, que es usada y usualmente
  controlada por sus desarrolladores, usada y aceptada entre los miembros de una
  comunidad virtual espec\'ifica.
\end{intro}