\begin{intro}
Usualmente cuando hablamos de dinero imaginamos un billete o una moneda. Sin
  embargo, cuando decimos dinero digital es m\'as complejo hacernos una imagen.
  Generalmente lo asociamos a pagos electr\'onicos, en los cuales no vemos el
  billete o moneda. El dinero digital exhibe propiedades similares al f\'isico,
  se diferencia en que permite transacciones instant\'aneas e intercambios
  entre distintos pa\'ises sin importar fronteras.

Una criptomoneda es un tipo de moneda digital, donde se utilizan medios
  criptogr\'aficos para asegurar las transacciones y para controlar la
  creaci\'on de nuevas unidades. La idea de utilizar herramientas
  criptogr\'aficas en monedas digitales surge como t\'opico de
  investigaci\'on en los a\~nos 80, cuando David Chaum \cite{chaum1983blind}
  introduce una nueva primitiva para realizar pagos imposibles de rastrear.

En el a\~no 2012 el banco central europeo \cite{bcentraleuro} define una moneda
  virtual como: \enquote{un tipo de moneda digital desregulada, que es emitida y
  usualmente controlada por sus desarrolladores, usada y aceptada entre los
  miembros de una comunidad virtual espec\'ifica}. Monedas virtuales
  seg\'un esta definici\'on existen desde hace tiempo, principalmente en
  comunidades de juegos.

Hasta el a\~no 2008 todas las monedas digitales conocidas, tanto en
  circulaci\'on como ya retiradas, compart\'ian una cualidad fundamental con
  el dinero f\'isico. Eran completamente controladas por una entidad central,
  tal como en el dinero f\'isico lo hace el banco central. Durante el mes de
  Noviembre del 2008 Satoshi Nakamoto publica ``\textit{A peer-to-peer
  electronic cash system}'' \cite{nakamoto2008bitcoin}, la primera moneda
  digital completamente descentralizada\cite{brito2013bitcoin}:
  \textit{Bitcoin}.

Tan solo unos meses despu\'es de su publicaci\'on ser\'ia puesto a
  disposici\'on de la comunidad un software implementando el protocolo
  propuesto. Empezaba as\'i a circular una moneda que cambiar\'ia radicalmente
  el escenario de las monedas digitales.


\end{intro}
