\section{Limitations}

\subsection{Number of participants}

Due the way the protocol is encoded into the transaction, the number of players
  must be equal to the number of possible outcomes on the bet's event.
During all this work we implicitly set this number to two, because usually bets
  are between two players, and events with binary outcomes are more simple.
However nothing prevents the protocol to be used with a larger number of
  players.

Because bitcoin limitation on the maximum size of each individual element of
  a script, the proposed protocol can not use more than 7 oracles when the
  number of players is two.

Bitcoin bounds the size of individual elements on the stack to 520 bytes.
We use 20 bytes hash for answer validation, and 33 bytes for the address
  to verify signatures. Therefore a lower bound on the size required for our redeem
  prize transaction is
  $ p\, \cdot\, (n\, \cdot\, 20\, +\, 33)$.
With two players ($p = 2$) and seven oracles ($n = 7$) it gives us a lower
  bound of 346 bytes.
The remaining 174 bytes are used on the transaction logic. This last value is
  due mostly because there are no loops. The code looks like an unrolled loop.

This equation also give us other important result: increasing the number of
  players decreases substantially the space available for oracles.

\subsection{Bet prize}

The cost of a protocol run is given by the bitcoin fees, and the number
  of oracles to use, as they define the transactions size and the payments
  to be made.
The cost does not depend at all on the amount of the prize for the winner,
  which makes relatively much more expensive bets of small prizes.
It will not make any sense to do a bet with a prize smaller than the cost
  of running the protocol.
