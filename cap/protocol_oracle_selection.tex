\section{The Protocol}

Our protocol is divided into two parts, where the second one is the actual bet
  and the first one is an optional step to select the oracles.

\subsection{First part: Oracle Selection}

The oracles are a key piece in the protocol, as they get to decide who get
  the prize money.
The first part of the protocol defines a way to select them in a trustless
  way.
The idea is quite simple, players selects from a list of oracles a subset
  to participate on its bet.
With a big enough list, selecting randomly from it reduces the
  chances from any of the participants to influence on the selection.
\colorbox{red}{citation required?}
However, there are a few key properties in our protocol to reduce the chance
  of influencing their selection.

\subsubsection{Oracle list}

In order to get a trustworthy list, we define a few key properties:
It must be a decentralized list; anybody willing to be an oracle can inscribe
  itself; and must be visible for both of the players.

As we saw in sub section \label{subsec:Blockchain}, we already have a
  public distributed database to store information.
We use the blockchain to keep the list of oracles, this provides tampering
  protection, a public database and a distributed source for the list.
In order to let anybody inscribe to be an oracle, the inscription is a simple
 transaction generated by the oracle sent to the blockchain.

\subsubsection{Oracle registration}
\colorbox{red}{Registration transaction}
In order to register themself into the list, oracles need to send a transaction
  to the blockchain.
This transaction must include the address to be used, this is the oracle
  identifier, and the protocol indentification.
We defined our own string as protocol indentification, but different protocols
  can use different strings to define other lists.

There is no required deposit for registration, however the transaction fee must
  be paid when sending the transaction.
Some may argue than a higher price to register an oracle will decrease the
  chances of an individual controlling the majority of the list.
If that is the case, increasing the cost by adding a required a unspendable
  output does not require any change in the transaction\colorbox{red}{link to the tx}, as the unspendable
  output already exist. Adding a deposit spendable by an address will require
  a new output, but the idea remain the same.

\subsubsection{Compiling oracle list}

There are a few parameters players must agree in order to select oracles from
  the blockchain list.
First they decide the period of time\footnote{Measured as a range of blocks in
  the blockchain.} they will consider oracles from.
Some participants might want to avoid recently registered oracles, as they might
  have an higher chance to be controlled by the other player.
Others might argue too old oracles are likely to be inactive, in order to avoid
  oracles registered long time ago.

Second they decide the list to get the oracles from, and if they want to filter
  out oracles that paid less than $b$ bitcoins on fees at registration time.
Finally they decide the number of oracles to use, also they can decide to
  select a few more than the required oracles, anticipating one or more of
  the selected oracles will not reply to the invitation.

Once they decide the filter and which blocks to use for retrieve the oracles,
  both players can compile the same list of availables oracles. Then they just
  need to decide which of them to use.

\subsubsection{Oracle selection}

Both players need to have the certainty the election from the list is random.
In order to achieve this property we use a protocol originally proposed to flip
  coins over the phone\cite{blum1983coin}.
Today this algorithm is mostly known as ``Coin Tossing'', and lives in a
  subfield of cryptography called Multi Party Computation.
Multi party computation, or secure multi party computation aims to provide
  protocols for computing public functions and gets its results while
  participants keeps their input private.

The idea of the Coin Tossing we use is to get a random bit, as neither of the
  players trust the other to select the bit randomly, both players select a bit
  and the they XOR it with the other one.
This way, does not matter how the other bit was chosen, the result is random.
There is one important restriction when using this protocol, the bit must be
  chosen before knowing the value of  the other, otherwise if one bit is known
  the second one can be selected in order to get the desired outcome.
If we were physically together we would write down the bit in a paper, wait for
  the other player to write his and then reveal both bits and perform the XOR.
However we would like to run this algorithm through the phone or in this case
  the computer.
The idea is the same, but instead of writting down into a paper, players
  ``commit'' to the value they just chose randomly by sending to the other
  player a ``commitment''.
This commitment binds the player to the value calculated, without revealing it.
Once both players receive the other's commitment, they send the bit they chose.
Both check the commit with the commitment received previously, and if they
  match, the protocol outputs a random bit.
Otherwise, a player tried to cheat and the protocol aborts as there is no way to
  calculate a random bit.

If we have a list both players agree with, and we can also produce random bits,
  selecting a number of oracles from the list is a trivial exercise.


