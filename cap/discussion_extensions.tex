\section{Extensions}

If we step out a little bit, the proposed protocol uses paid oracles to get a
  binary answer about an event outside the blockchain.
This is useful not only for betting on some outcome.
The oracles are oblivious to the protocol in which their answer is used; they
  get paid anyway.
Further applications of the protocol can be generalized from our proposal.
Resolution of contractual disputes is an interesting topic where parties agree
  to use arbitrator(s) to decide two conflicting interpretations of a contract,
  for example.
In this example, oracles take part in the protocol in a role that is more
  similar to a judge than to an oracle.
One major difference is that one should not care whether a given judge's
  opinion matches the others' opinion, but instead that the judge provides
  his/her independent opinion.
So, it would make sense to paid the oracle for providing an answer instead of
  providing the same that of the majority.
To implement the proposed protocol from our protocol we can simply remove the
  output with the penalty on a wrong answer.

Other possible extension would be to move beyond untrusted and randomly
  selected oracles to decide the bet.
It may be interesting to consider semi-trusted oracles, as ``oracles as a
  service'' could be a realistic business option.
This might require different types of payments to the oracles; semi-trusted
  oracles may charge more for participate than random ones from the list.
Our protocol was proposed with the assumption all oracles are equivalent,
  therefore they get paid equally, but it also may be extended to support
  different rewards and penalties per oracle.
