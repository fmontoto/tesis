\section{Incentives}

The ulterior motive for participating in a betting protocol is to make money, so
  we assume each player is driven by this logic.
All of their actions are consistent with this goal, earn money.
The protocol is designed with this assumption in mind and, in the following
  paragraphs, we discuss how the monetary incentives encourage participants to
  behave properly.

\subsection{Players}

The first thing to considerate is that the protocol runs on top of bitcoin
  transactions, which have a cost (fee).
Once the first transaction is placed, at least one player will not recover its
  money: if the protocol does not finishes with a resolution, both will lose
  some money; if there is a winner, such entity will get earnings and the other
  will lose the money involved.

One strategy to maximize the option to win is to control the oracles.
The first phase of the protocol and the second's player enforcement
  prevent this.
However there is another way to control the oracles, payments can be made to
  influence into the oracle's answer.
Bribing the oracles is discussed into the Section~\ref{subsec:inc_oracles}.

Simply causing the other player to lose money could be a motivation to some
  players, even if such action does not mean an earning for themself.
As payments on timeouts and fees are equally distributed in the transactions,
  aborting the protocol at some point will mean an equal lose of funds for both
  players. If a player is willing to make the other one lose an amount of money,
  it will cost him the same amount.
Other possible motivation could be to deprive the other player of its funds.
But again for the same reason mentioned above, this will mean the player
  performing the attack should lock the same amount of money for the same period
  of time.

A malicious player can impair monetarily the other player, but it will not be
  for free, as it will cost the same amount of money for the malicious player.
Finally, one player can make the other player lose money, but only the small
  amounts to be used as fees.
When it comes to funds withholding, the amount may end up being all the money
  involved, but it will certainly not be lost, just locked during the timeout
  associated to the bet.

\subsection{Oracles} \label{subsec:inc_oracles}

As the players, oracles also seek to maximize their earnings.
In principle, for the protocol that means the oracle must give the correct
  answer to collect its payment.
But, there is an option to increase the earnings outside the protocol. Receiving
  money from a player to change their vote can give the oracle more money than
  answering correctly, as the incentive to change its vote can be bigger than
  the payment.
A modified version of the bet transaction can used to do these payment, the
  player willing to pay the bribe can set the output to be spent with the answer
  he expects.
So, in order to get the bribe, the oracle must reveal the answer the player is
  paying for.
This make the problem even bigger, as no trust is required between a player and
  the oracle in order to cheat and change the answer.

This problem comes from the fact there is not source of truth accessible in the
  blockchain.
In fact this is the main problem the protocol tries to solve.
The payment for answering correctly goes for the oracles that answer as the
  majority of them, not the ones that answer the truth, simply because that is
  the protocol truth.
Bribing a majority of the oracles gives the oracles the bribe and also the
  payment.
And this is the only useful bribe for the player, to change a minority of the
  answers does not gives him any benefit.

A way to mitigate this problem is to give the oracles certain reputation based
  on past behavior.
This gives the oracles the incentive of behave properly in order to get long
  term earnings, as accepting bribes will erode their chances to be selected as
  oracles again.
Players must consider this incentive when choosing oracles, for example,
  selecting some kind of reputation in the hope this approach decreases the
  chance of selecting malicious oracles.
