\section{Communication}

When the protocol is explained many things are get by granted, we just say
  oracles and players exchange communication between them with not further
  discussion.
When implementing this protocol the communication between participants can
  not be ignored, in this section we discuss a proposed model of communication
  for the protocol.

In Bitcoin communication in the protocol is not encrypted neither
  authenticated, the Bitcoin's protocol has incentives and cryptographic
  protection that make this possible.
The protocol relies heavily on the Blockchain, so we take advantage of all these
  features and no communication requieres encryption nor authentication.

There are two required things both players needs to know in order to start the
  protocol, the other player's Bitcoin address and a communication method.
A communication method besides the blockchain is required in order to discuss
  and get to an agreement in the bet parameters. This communication is required
  to be bidirectional.

The first step is to chat and decide the bet, if this communication is not
  secured, the \textit{Bet Promise} transaction works as authentication method.
Because at signing it, each player proves it control the private key
  corresponding to its address.

When an oracle sees its id in the blockchain it needs to get the full bet
  description to decide whether to participate or not, as the full description
  hash is in the \textit{Bet Promise} transaction, it can be retrieved using an
  insecure channel.
If the oracle decides to participate it need to send the \textit{Oracle
  Inscription} transaction to the players, this transaction is signed with the
  oracle's private key, players must check it matches the address they selected.
Oracle also sends the transaction script, as a P2SH only contains the hash of
  it, players are to check the transaction is as the protocol requires.

After all the oracles sent their transactions, players have all they need
  to create, sign and send the Bet transaction to the blockchain.
No further communication off the blockchain is required from now on if every
  oracle behaves correctly.
However, if a penalty is to be charged to an oracle,\foonote{Not responding
  on time for example.} communication off the blockchain is required to
  build and sign the transaction charging the penalty to the oracle.


\subsection{Secured communication} \label{subsec:secured_comm}

As stated above, there is no need for secured communication in the protocol,
  however it might be desired for some participants, here we propose how to
  start a secure communication with no extra previous knowledge using a peer
  to peer channel.

Both players known each other's Bitcoin address, so the first step is to
  reveal to the other the address' public key using an insecure channel.
This is verifiable by the receiver, as the address is the hash of the public
  key.
Then they generate and send a random string in the insecure channel.
After this, using the public key of the other player, a secure channel id and
  credentials to use it and the received random string are encrypted and sent
  using the insecure channel.
This message is decrypted and a equality check is done, the random string
  received must match the one sent at the begining, this prevent replay
  attacks.

At this point both players know the secure channel and have the credentials
  to connect to it, secure communication can be started between them. This
  step is also reproducible with the oracles, nothing changes as the oracle's
  address is also known.

\subsection{Channel}

For test purposes we used a plain TCP connection between peers, and also
  a secure channel was implemented using
  CurveZMQ \footnote{http://curvezmq.org/}.
An authentication and encryption protocol on top of TCP that uses elliptic
  curves cryptography to protect the packages sent in the wire.

Establishing a peer to peer connection as channel might reveal more information
  than participants might want.
A third party communication channel can be used to obscure our id from the other
  players, but it might be known by the third party controlling the channel.
Several options can be used to obscure real id when communicating, using private
  navigation as the one provided by the Thor Network
  \footnote{https://www.torproject.org/} can help when using a third party
  channel.
Other approach is to use inherently anonymous communication protocols, as the
  one Orisi uses, BitMessage\cite{warren2012bitmessage}.
Which solution to use will vary from case to case and how difficult to track
  the participant wants to be.

