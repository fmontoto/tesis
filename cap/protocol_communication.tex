\section{Communication}

When the protocol is explained, we have left many things on the intuitive level:
  we just said oracles and players exchange data between them with not further
  discussion.
When implementing this protocol the communication between participants can
  not be ignored.
In this section, we discuss a proposed model of communication for the protocol.

In Bitcoin, all protocol communication is not encrypted neither authenticated.
The Bitcoin's protocol has incentives and cryptographic protection that make
  this possible.
The protocol relies heavily on the Blockchain, so we take advantage of all these
  features and we add no encryption nor authentication to the protocol
  communication.

In order to start the protocol, players are required to know two things about
  each other: Each other player's Bitcoin address and a specific communication
  method.
A communication method besides ``the blockchain'' is required in order to
  discuss and agree on the bet parameters.
This communication must be bidirectional.

The first step is to chat and decide the bet, if this communication is not
  secured, the \textit{Bet Promise} transaction works as authentication method.
This is because, at signing it, each player proves it control the private key
  corresponding to its address.
This way, each player commits to the data the transaction holds.

When an oracle sees its \texttt{id} in the blockchain it needs to get the full
  bet description to decide whether to participate or not, as the full
  description hash is in the \textit{Bet Promise} transaction.
This description can be retrieved using an insecure channel.
If the oracle decides to participate it needs to send the \textit{Oracle
  Enrollment} transaction to the players, which is signed with the oracle's
  private key.
Then, players must check it matches the address they selected.
Oracle also sends the transaction script, as a P2SH only contains the hash of
  it, players must check that the transaction is consistent with what the
  protocol requires.

After all the oracles sent their transactions, players have all they need
  to create, sign, and send the Bet transaction to the blockchain.
No further communication off the blockchain is required from now on if every
  oracle behaves correctly.

\subsection{Secured communication} \label{subsec:secured_comm}

As stated above, there is no need for secured communication in the protocol,
  however if required for some participants, we now propose a mechanism to
  start a secure communication with no extra previous knowledge, simply using
  a peer to peer channel.

Both players known each other's Bitcoin address, so the first step is to
  reveal to the other the address' public key using an insecure channel.
This is verifiable by the receiver, as the address is the hash of the public
  key.
Then they generate and send a random string in the insecure channel.
After this, using the public key of the other player, the following
  information is encrypted and signed.
A secure channel id, credentials to establish the secure channel, and the
  received random string. This encrypted values are sent using the insecure
  channel.
This message is decrypted and a equality check is done, the random string
  received must match the one sent at the beginning, to prevent replay
  attacks.

At this point both players know the secure channel and have the credentials
  to connect to it, secure communication can be started between them. This
  step is also reproducible with the oracles, nothing changes as the oracle's
  address is also known.

\subsection{Channel}

For test purposes we used a plain TCP connection between peers.
A secure channel was implemented using
  CurveZMQ\footnote{http://curvezmq.org/}, an authentication and encryption
  protocol implemented on top of TCP that uses elliptic curves cryptography to
  protect the communication.

Establishing a permanent peer to peer connection (a channel) might reveal more
  information than participants might want.
A communication channel via a trusted third party can be used to obscure our id
  from the other players, but this opens the door to abuse by the (trusted)
  third party controlling the channel.
Several options can be used to obscure real id when communicating.
Using private navigation as the one provided by the Thor network\footnote{%
    https://www.torproject.org/} can help when using a third party channel.
Another approach is to use other anonymous communication protocols, as the
  one Orisi uses, BitMessage \cite{warren2012bitmessage}.
Which solution to use will vary from case to case and the degree of anonymity
  the participants desire.

