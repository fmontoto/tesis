\section{Oracles}

The first issue to solve when making decisions over events in the world is how
  to choose who tracks and defines the outcome of given event.
Usually we get information about events from a variety of sources.
From television, an internet portal, or our eyes among many others. Any protocol
  willing to make decisions over events needs a source for those events.

In order to keep the protocol decentralized we define the source of truth to be
  a majority from a set of entities, called oracles.
The answer for questions asked within the protocol is decided by the oracles
  voting.
Using this scheme the decision does not rely on a centralized entity, but in a
  group of them.

Oracles are rewarded when provide the correct answer to resolve the bet.
We define the correct answer as the one given by at least $m$ of the $n$ oracles
  where $\lfloor \frac{n}{2} \rfloor < m$.
When providing the incorrect answer oracles do not get paid, and when giving
  both answer they get penalized because its misbehavior. This gives strong
  economic incentives to the oracles to answer as they expect the other ones
  are going to answer.
A discussion on how this incentives influence oracles behavior is available
  in Section~\ref{subsec:inc_oracles}.
