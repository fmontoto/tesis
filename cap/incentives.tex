\section{Incentives}

The ulterior motive for participating in the protocol is to win money, so we
  assume each player is driven by this.
All of their action aims to this goal, win money. The protocol is designed with
  this assumption in mind and in the following paragraphs we discuss how the
  monetary incentives align participants to have a proper behaviour.

\subsection{Players}
The first thing to considerate is that players are paying fees for every
  transaction they send to the blockchain, in equal proportions.
So, it is impossible the protocol can be aborted and both players get its
  money back.
Once the first transaction is placed, at least one player will not recover its
  money: If the protocol does not finishes with a resolution, both will lose
  some money; If there is a winner the other will lose everything.

One option to maximize the money won is to control the oracles.
The first phase of the protocol and the second's player enforcement
  minimize the chances of selecting them.
However there is another way to control the oracles, payments can be made to
  influence into the oracle's answer.
Bribing the oracles is discused into the subsection \ref{subsec:inc_oracles}.




\subsection{Oracles} \label{subsec:inc_oracles}

As the players, oracles also try to maximize their earnings, within the protocol
  that means to give the correct answer and collect its payment.
But, there is an option to increase the earnings outside the protocol. Receiving
  money from a player to change their vote can give the oracle more money than
  answering correctly, as the incentive to change its vote can be bigger than
  the payment.
A modified version of the bet transaction can used to do these payment, the
  player willing to pay the bribe can set the output to be spent with the answer
  he expects.
So, in order to get the bribe, the oracle must reveal the answer the player pay
  to get.
This make the problem even bigger, as no trust is required between a player and
  the oracle in order to cheat and change the answer.

This problem comes from the fact there is not source of truth accessible in the
  blockchain, in fact this is the problem the protocol tries to solve.
The payment for answering correctly goes for the oracles that answer as the
  majority of them, not the ones that answer the truth, simply because that is
  the protocol truth.
Bribing a majority of the oracles gives the oracles the bribe and also the
  payment.
And this is the only useful bribe for the player, to change a minority of the
  answers does not gives him any benefit.

A way to mitigate this problem is to give the oracles certain reputation based
  on past behavior.
This gives the oracles the incentive of behave properly in order to get long
  term earnings, as accepting bribes will erode their chances to be selected as
  oracles again.
\colorbox{red}{How to track bad behavior?}
