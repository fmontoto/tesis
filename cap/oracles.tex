\section{Oracles}

The first issue to solve when making decisions over events in the world is to
  define who track and define the outcome of said event.
In our day to day we get information about events from a variety of sources. The
  television, an internet portal, our eyes among many others. Any protocol
  willing to make decisions over events needs a source for those events.

In order to keep the protocol decentralized we define its data source as a set
  of entities, called oracles. The decision is made by the oracles voting on
  the outcome of the event. In this scheme the decision does not rely on a
  centralized entity, but in a group of them.

Oracles are rewarded when provide the correct answer to resolve the bet.
We define the correct answer as the one gave by at least $m$ of the $n$ oracles
  where $\lfloor \frac{n}{2} \rfloor \leq m$.
When providing the incorrect anwswer oracles do not get paid, and when giving
  both answer they get penalized because its misbehavior. This gives strong
  economic incentives to the oracles to answer as they expect the other ones
  are going to answer. \colorbox{red}{Explicar enseguida lo de los incentivos?}
