En primer lugar a mis padres, que me han apoyado desde siempre e hicieron
  posible que yo llegara a esta instancia.
Sin su amor e incondicional apoyo, este trabajo nunca hubiera visto la luz.

A la Ke\~na que me recibi\'o en su hogar durante mis los largos a\~nos
  universitarios. El llegar a un hogar familiar a estudiar sin duda hizo
  m\'as llevadera la universidad. Gracias por todo el cari\~no y paciencia
  estos a\~nos.

A mi profesor gu\'ia, Alejandro Hevia, quien me acompa\~n\'o, y gui\'o
  a trav\'es de este proceso. Gracias por la flexibilidad en el trabajo
  y las revisiones minuciosas.

A Fernanda Ahumada, que me apoy\'o, ayud\'o y presion\'o para terminar de
  manera pronta mi tesis. Gracias por la comprensi\'on y por estar ah\'i en los
  momentos de mayor estr\'es. Sin tu amor (y tu ayuda para descifrar las
  correcciones) todo hubiera sido m\'as dif\'icil.

A todos con quienes compart\'i en NicLabs, especialmente a su director, Javier
  Bustos por el grato ambiente laboral y haberme empleado de manera flexible
  permiti\'endome trabajar en mi tesis.

A Sandra Ga\'ez, Ang\'elica Aguirre y a los integrantes de mi comisi\'on, por
  permitir y hacer lo posible para poder defender este trabajo antes del fin
  del a\~no acad\'emico, a pesar del poco tiempo disponible.

Finalmente a Cristi\'an y Dietrich, con quienes compart\'i un hogar mientras
  escrib\'i la mayor parte de mi tesis. Gracias por propiciar espacios
  y tiempos al no obligarme a jugar todos los d\'ias, sin duda fue crucial
  para poder escribir.
