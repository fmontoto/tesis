Cada vez que realizamos una apuesta, implícitamente confiamos en que la
  contraparte nos pagar\'a en caso de perder.
Por esta raz\'on al apostar con alguien que no conocemos o no confiamos,
  no lo hacemos de manera directa, sino que utilizamos un casino u otro lugar de
  apuestas.
Confiamos en que la ley, o la reputaci\'on del lugar forzar\'a la correcta
  ejecuci\'on de la apuesta.\\
\noindent
 Las criptomonedas han hecho posible el guardar y transferir dinero de manera
   f\'acil y sin una autoridad central.
Dada su reciente popularidad, muchor procesos descentralizados han sido
  propuestos e implementados en ellas.
Por ejemplo, podemos apostar a un dado, carta o ruleta. No es necesario confiar
  en ninguna autoridad central durante el proceso.\\
\noindent
En este trabajo proponemos un protocolo sobre Bitcoin para realizar apuestas
  sobre eventos en el mundo real, sin requerir una entidad centralizada ni
  confianza en ninguna entidad individual.
Nuestro protocolo hace uso de un grupo de \textit{or\'aculos} para proveer
  resultados de eventos que toman lugar en el mundo real en la blockchain de
  bitcoin.
Cuando un umbral definido the or\'aculos reportan el mismo resultado para un
  evento, el dinero de la apuesta pasa al control del ganador.
El protocolo est\'a dise\~nado para mover el dinero directamente hacia el
  ganador, los or\'aculos en nig\'un momento tienen control arbitrario sobre
  el dinero, solo pueden escoger como destino del dinero uno de los
  jugadores en la apuesta.\\
\noindent
Los incentivos econ\'omicos en el protocolo est\'a dise\~nados para premiar
  el correcto comportamiento de los participantes.
Los or\'aculos reciben un pago por ayudar a resolver la apuesta solo en el
  caso de proveer la respuesta correcta, donde la correctitud la define
  la mayor\'ia de ellos.
Una vez que la apuesta es decidida, ambos jugadores en conjunto bloquean
  el dinero de \'esta y le entregan el control a la mayor\'ia de los or\'aculos
  para que decidan de entre ellos el recipiente del premio.
En caso de que la apuesta no se resuelva, porque los or\'aculos no proveen
  una respuesta, el dinero vuelve a cada uno de los jugadores.\\
\noindent
Utilizando el lenguaje de programaci\'on que Bitcoin provee, el protocolo
  garantiza justicia y una ejecuci\'on correcta desde el momento en que
  la apuesta es definida. Incluso cuando una minor\'ia de los or\'aculos y
  un jugador deshonesto est\'an participando.
Antes de definir la apuesta, or\'aculos y un jugador deshonesto pueden
  causar un da\~no econ\'omico limitado a un jugador honesto.
Sin embargo este ataque tiene un costo igual al da\~no causado, por lo que
  no produce una ganancia neta al atacante.\\
\noindent
El protocolo fue implementado y testeado en la red de pruebas de Bitcoin,
  utilizando el cliente oficial para verificar las transacciones.
Nuestra implementaci\'on est\'a actualmente disponible como proyecto de
  c\'odigo abierto.\\
\noindent
Los incentivos econ\'omicos para los participantes y los costos de
  ejecutar este protocolo son discutidos en profundidad.
Creemos que este trabajo es una contribuci\'on original en la forma en que
  se utilizan eventos en el mundo real como input en la blockchain.
Nuestro protocolo puede ser f\'acilmente extendido para resolver
  disputas contractuales o situaciones similares donde decisiones
  subjetivas pero justas son requeridas, de una forma descentralizada.
