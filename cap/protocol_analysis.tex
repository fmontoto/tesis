\section {Cost analysis} \label{sec:cost_analysis}
\newcommand\numoracles[0]{7}%
\newcommand\feeval[0]{15}%
\newcommand\bitcoinusd[0]{3800}%
\newcommand\txcost[2]{\calculatecosts{#1}{#2}{\numoracles{}}{\feeval{}}}%
\newcommand\usdcost[1]{\satoshitousd{#1}{\bitcoinusd{}}}%

Most of the proposed bitcoin protocols in the literature make the simplified
  assumption that transaction fee is 0, as this was the cost of it for a long
  time.
However as bitcoin use has increased, transactions are not free
  anymore\footnote{We can still send transactions with no fees to the
    blockchain, but it might take forever for them to get into it. As miner will
    prioritize transactions with fee to collect.}.
That is why we decided to explicitly consider the fees in the protocol
So in what follows we do the analysis with all the costs, not only the oracle
  payment.

In Bitcoin, fees are charged by byte, therefore bigger transactions will pay more
  money as fee, no matter how much money they spend.

At the moment we write this work, transactions paying 120 satoshi\footnote{A
  satoshi is the smallest unit of bitcoin on the blockchain. It is a one hundred
  millionth of a single bitcoin ($1 \cdot 10^{-8}$).} usually gets into the next
  mined block.
That is not too bad for money transferences, but as we have some big transaction
  we will do calculations with a fee of \feeval{} per byte.
This usually will not get our transactions into the first block, but into the
  first 15, this means it can take up to 3 hours to get the transactions in
  the blockchain.
This is enough for an average case with enough time between timeouts.
Players in a tight schedule can expend more money in fees and submit
  transactions faster.

It is impossible to give an exact value for the size of each transaction, as
  addresses and signatures do not have a fixed size.
For this analysis, we use average size values, which consider a fluctuation of
  under 5\%.
Table~\ref{tab:tx_fees} has the size for the transactions used in the protocol.
Sizes are in bytes and we consider \numoracles{} oracles.

\begin{center}
    \begin{tabular}{|c|c|c|c|c|}
      \hline
        \textbf{Transaction} & \makecell{\textbf{Constant} \\ \textbf{size}} &
          \makecell{\textbf{Per oracle} \\ \textbf{size}} &
          \makecell{\textbf{Total} \\ \textbf{size}} &
          \textbf{Fee [satoshi]} \\
      \hline
        Oracle Registration & \txcost{239}{0} \\
      \hline
        Bet Promise & \txcost{1267}{65} \\
      \hline
        Oracle Enrollment & \txcost{776}{0} \\
      \hline
        Bet & \txcost{617}{445} \\
      \hline
      \hline
        Player redeem prize & \txcost{511}{150} \\
      \hline
        Oracle redeem payment & \txcost{355}{0} \\
      \hline
        Oracle redeem undue & \txcost{283}{62} \\
      \hline
        Oracle redeem two answers & \txcost{323}{0} \\
      \hline
        Player redeem wrong answer & \txcost{338}{70} \\
      \hline
        Player redeem two answers & \txcost{373}{0} \\
      \hline
        Player redeem oracle doesn't answer & \txcost{439}{0} \\
      \hline
    \end{tabular}
    \captionof{table}{Transactions size and fee.}
    \label{tab:tx_fees}
\end{center}

The first four transactions are the ones detailed in the previous section.
These transactions however are not enough to make the complete cost analysis
  of a protocol execution.
Redeem transactions are required to put back the money into personal accounts,
  so we append them.
They are explained next:

\begin{itemize}
    \item Player redeem prize: takes the earnings of the winner into an address
        the player controls.
    \item Oracle redeem payment: this transaction includes the vote to select
        the winner and allows the oracle to retrieve him/her reward.
    \item Player redeem oracle doesn't answer: can be submitted by both players
        when an oracle doesn't answer. It charges the oracle with the associated
        penalty.
    \item Player redeem wrong answer: it is the transaction where the winning
        player can charge oracles that didn't answer correctly.
    \item Oracle redeem undue: it is submitted by each oracle that does not
        provide an incorrect answer (this includes failing to provide an
        answer).
    \item Player redeem two answers: it is a transaction that can be submitted
        by any player when a given oracle gave two answers to collect the
        penalty associated.
    \item Oracle redeem two answers: it can be submitted by any oracle that
        didn't give two answers; it recovers the penalty for this behavior back
        to the oracle.
\end{itemize}

For a successful run of the protocol, where there is a winner and every oracle
  behaves properly. Table~\ref{tab:costs}  summarize the costs incurred by both
  players.

\newcommand\totalcost[2]{\totalcostimpl{#1}{#2}{\numoracles{}}{\feeval{}}}%
\begin{center}
    \begin{tabular}{|c|c|}
        \hline
            \textbf{Item} & \textbf{Cost [satoshi]} \\
        \hline
          Bet Promise fees & -\totalcost{1267}{65} \\
        \hline
          Oracle Enrollments fee & -\totalcost{0}{776} \\
        \hline
          Bet transaction fee & -\totalcost{617}{445} \\
        \hline
          Oracles payment & -\numoracles{}\ $\cdot$ [Oracle Payment] \\
        \hline
          Oracle first transfer & -c \\
        \hline
          Player redeem prize & -2 $\cdot$ \totalcost{511}{150} \\
        \hline
          \textbf{Total} & -\numoracles{}\ $\cdot$ [Oracle Payment] - c - \num{210120} \\
        \hline
    \end{tabular}
  \captionof{table}{Successful run, costs for the players}
  \label{tab:costs}
\end{center}

The number of oracles is \numoracles{}, as set above.
The value of c and [Oracle Payment] are parameters decided by the players,
  below we propose values for this constants and discuss about the tradeoff on
  setting this values.

In table~\ref{tab:oracle_costs} we summarize the cost and earnings for an oracle
  that behaves correctly in a protocol execution.

\begin{center}
    \begin{tabular}{|c|c|}
        \hline
            \textbf{Item} & \textbf{Cost [satoshi]} \\
        \hline
          Initial deposit & -[Registration] - [Oracle Payment] - [Two Answers Penalty] \\
        \hline
          Payment & [Oracle Payment] \\
        \hline
          Redeem payment fee & -\totalcost{355}{0} \\
        \hline
          Undue deposit & [Oracle Payment] \\
        \hline
          Undue redeem fee & -\totalcost{283}{62} \\
        \hline
          Two answers deposit & [Two Answers Penalty] \\
        \hline
          Two answers redeem fee & -\totalcost{373}{0} \\
        \hline
        \textbf{Total} & [Oracle Payment] - [Registration] - \num{21675} \\
        \hline
    \end{tabular}
    \captionof{table}{Successful run, oracle cost and earning}
    \label{tab:oracle_costs}
\end{center}

The constant c captures  transfer made from the players to the
  \textit{Oracle Enrollment} bet.
Its objective is to mitigate the oracle risk of accepting to participate in the
  bet.
Namely, if an oracle sends the \textit{Oracle Enrollment} and the bet does not
  take place.
In this case, the oracle will get back all its deposit plus $c/n$ minus
  \fee{OracleEnrollment}.

Players might decide, in order to incentive oracles to participate, to eliminate
  the risk of losing money in this situation for oracles.
So the players can set c such that $c/n > \fee{OracleEnrollment}$.
Nonetheless, players might decide to make $c = 0$ because this simplifies the
  transaction and saves money in fees.

The [Registration] constant is an option to require the oracle to lock more
  money while participating in the bet.
Making this constant big could help to have more committed oracles, as they
  have more at stake in the execution.
Players have the option to set it at zero, but this will not save any fees.
Probably the most important constant in this analysis is the [Oracle Payment],
  it's the most expensive cost for the players and the way oracles earn money
  by participating.

As the tables show there is no dependency between the protocol costs and the
  amount of the bet.
The cost for paying oracles is directly proportional to the number of them, and
  how much is each one getting paid.
And the transaction cost depends on the number of oracles, the size of the
  transactions, and a per byte fee.

\subsection{Concrete costs}

So, how much does it cost a run of the protocol?
At the moment this work was written bitcoin was being exchange by USD
  \bitcoinusd{}.
If we replace this number in the first table we get the cost of the protocol for
  the players:
\begin{center}
    \begin{tabular}{|c|}
        \texttt{7 * [Oracle Payment] + c + USD 7.98}
    \end{tabular}
\end{center}

As the \textit{Oracle Registration} transaction fees correspond to
  % The macro has a bug somewhere, mbox to prevent the new line.
  \mbox{\totalcost{239}{0}} satoshis, we set the parameter $c/n$ to 5000, a
  little bit more than the cost of the fees.
This gives a value of $c = 35\,000$, so the cost of c is USD 1.33.

The discussion about the value for the oracle payment is less technical, and
  probably each person reading this work might have a different optimal value
  in mind.
We have simplified this payment to be equal for every oracle, but players
  might decide some oracles are more expensive than other ones.
Oracles with more history or reputation might be worthy of a bigger payment.
The amount being wagered is also an important factor to determine this number,
  if there is a really big amount of money at stake we might be willing to
  spend more money on this item, as this can decrease the chance of the oracles
  taking a bribe.

For this calculation we have defined a [Oracle Payment] of \num{200000}
  satoshis, or USD $7.60$. Simplifying the [Registration] to 0, running the
  protocol cost the users USD $62.51$.
In order to participate in the bet, each oracle needs to put a deposit of
  USD 31.25, and if behaves properly at the end of the bet will earn USD 6.40

To summarize, if player are willing to bet USD 1000, they are required to start
  with USD 1031.25, the winner would have win USD 968.75 at the end, and the
  loser would have spent 1031.25.
