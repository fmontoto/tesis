\section {Cost analysis}
\newcommand\numoracles[0]{7}
\newcommand\feeval[0]{15}
\newcommand\txcost[2]{\calculatecosts{#1}{#2}{\numoracles{}}{\feeval{}}}

Most of the bibliography read simplifies the fee transactions to be 0,
  as this was the the cost of it for a long time.
However as bitcoin use has increased, transactions are not free anymore
  \footnote{We can still send transactions with no fees to the blochain, but it
    might take forever for them to get into it. As miner will prioritize
    transactions with fee to collect.}, that's why we kept the fees in the
  explanation.
So we can do the analysis with all the costs, not only the oracle payment.

In Bitcoin, fees are charged by byte, therefore bigger transactions will pay more
  money as fee.
No matter how much money they spend.

At the moment we write this work, transactions paying 120 satoshi\footnote{A
  satoshi is the smallest unit of bitcoin on the blockchain. It is a one hundred
  millionth of a single bitcoin ($1 \cdot 10^{-8}$).} usually gets into the next
  mined block.
That is not too bad for money transferences, but as we have some big transaction
  we will do calculations with a fee of \feeval{} per byte.
This usually will not get our transactions into the first block, but into the
  first 15, this means it can take up to 3 hours to get the transactions in
  the blockchain.
This is enough for an average case with enough time between timeouts.
Players in a tight schedule can expend more money in fees and submit
  transactions faster.

It's impossible to give an exact value for the size of each transaction, as
  addresses and signatures do not have a fixed size, for the analysis we use
  average the size values, the fluctuation is under 5\%.
Table \ref{tab:tx_fees} has the size for the transactions used in the protocol,
  sizes are in bytes and we calculate the total using \numoracles{} as the number
  of oracles:

\begin{center}
    \begin{tabular}{|c|c|c|c|c|}
      \hline
        \textbf{Transaction} & \makecell{\textbf{Constant} \\ \textbf{size}} &
          \makecell{\textbf{Per oracle} \\ \textbf{size}} &
          \makecell{\textbf{Total} \\ \textbf{size}} &
          \textbf{Fee [satoshi]} \\
      \hline
        Oracle Registration & \txcost{239}{0} \\
      \hline
        Bet Promise & \txcost{1267}{65} \\
      \hline
        Oracle Inscription & \txcost{776}{0} \\
      \hline
        Bet & \txcost{617}{445} \\
      \hline
      \hline
        Player redeem prize & \txcost{511}{150} \\
      \hline
        Oracle redeem payment & \txcost{355}{0} \\
      \hline
        Oracle redeem undue & \txcost{283}{62} \\
      \hline
        Oracle redeem two answers & \txcost{323}{0} \\
      \hline
        Player redeem wrong answer & \txcost{338}{70} \\
      \hline
        Player redeem two answers & \txcost{373}{0} \\
      \hline
        Player redeem oracle doesn't answer & \txcost{439}{0} \\
      \hline
    \end{tabular}
    \captionof{table}{Transactions size and fee.}
    \label{tab:tx_fees}
\end{center}

The first four transactions are the ones detailed in the previous section,
  however this transactions are not enough to make the complete cost analysis
  of a protocol execution.
So we append the redeem transactions, as they are required to get the money
  into complete user control form the protocol.



The payment of the oracle's is not the only cost of this protocol.
There is a significant number of transactions in the protocol, as there is a fee
  by each transaction in the Blockchain, this makes the protocol more expensive.
This is a problem for small bets, the presented protocols is prohibitively
  expensive for bet of just a few dollars. At least with the current fee costs
  of bitcoin.

