\section{Previous work} \label{sec:previous_work}

There are several attempts to provide information to the blockchain from the
  outside.
One possibility is using  ``Distributed Oracles'', where the data is channeled
  through a group of third party participants.
Another option is using the so called ``Data Feeds'', where the data is provided
  by a centralized party, using some cryptographic techniques to authenticate
  the data.

We also add ``Trustless Distributed Casino'' as they provide a solution to a
  similar problem.
The perform trustless bets, but not  on real world events.
The service is limited to bets on random events.

\subsection{Distributed oracles: Orisi}
Orisi \cite{orisiwhitepaper} is a distributed system for bitcoin smart
  contracts that relies on multiple oracles to bring information from outside
  of the blockchain.
It allows users to transfer money from one address to another when a condition
  is met.

The way the system works is as follows:
In the first step both players choose 7 oracles who will decide the winner of
  the bet.
These oracles are chosen from ``The Oracle List'', a curated list with oracles.
A multisignature address is then generated to store the money while the bet
  takes place.
A multisignature address is defined by \texttt{m > 1} different addresses and a
  required number \texttt{n} (\texttt{n} < \texttt{m}) of them are required to
  sign.
A valid signature for a multisignature address is generated by using at least
  \texttt{n} out of these \texttt{m} addresses.

The multisignature address generated will store the money until the oracles
  decide the winner.
To avoid the oracles stealing the money the multisignature transaction includes
  the address of the receiver.
Therefore we want a $1$ + ($n$ of $m$), where the extra signature is
  from the receiver.
As this kind of transaction is not considered
  standard\footnote{Non standard transactions are recognized as valid
  by everybody at the bitcoin network, but only a small fraction of
  miners will mine them.}.
 With this configuration the oracles are
  not able to move the money by themselves, and at least one signature
  from the receiver is required.

\subsection{Trustless distributed casino: Winsome.io}

In may 2016 Rouleth \cite{winsomeio} was launched as a distributed application
  on the ethereum network.
It offered players a ``provably-fair'', real money roulette.
Soon after ``BlockJack'', another game,  was launched using the Ethereum
  network.
It is the first playable blackjack game on the Ethereum mainnet.

Winsome.io is the instance where these games are enclosed and offers unique
  advantages over traditional casinos (physical and virtual), like no need to
  trust on it, and complete control over the funds the entire time while
  playing.
It does work in a distributed fashion using smart contracts, publicly available
  for everyone's scrutiny.

Winsome.io provides its users trustless gambling over random events.
By using the ethereum network as backend.
It has been quite successful and one of the most popular decentralized
  applications on the Ethereum Network.

\subsection{Secured data feeds}
A secured data feed is a centralized stream of data, where a trusted entity
  ensures the correctness of the data provided by it.
It is usually protected from tampering by cryptographic algorithms.
Although this is a centralized data source, using many techniques or providers
  might be combined to get a decentralized source.
Several services provide this feeds as a service to use on smart contracts in
  the blockchain.

\subsubsection{Oraclize}
Oraclize \cite{oraclizeit} provides an interface for using data fetched from a
  web site in the ethereum blockchain, working with arbitrary URLs or queries
  in certain web services, as ``Wolfram Alpha''\footnote{Wolfram Alpha is a
  knowledge engine able to answer queries rather than provide links to data
  sources, as a search engine does.}.
It provides an Authenticity Proof of the data gathered, so the user can check
  that the data provided by the interface was generated by the source and had
  not been tampered.

The most common authenticity proof is the ``TLSnotary Proof'', which leverages
  in a feature from the TLS protocol.
Oraclize provides a signed attestation that a proper TLSnotary proof did occur,
  on its server.
This implementation is implicitly trusting in Oraclize and the web page to
  provide the right data.

\subsubsection{Town Crier}
Town Crier \cite{zhang2016town} is an authenticated data feed system for
  the ethereum blockchain.
As oraclize it works as a bridge between web feeds, and the blockchain.
It uses an Intel technology called ``Software Guard Extensions''
  \cite{costan2016intel}, than provides some execution guarantees of the software
  executed by hardware protected areas.
This protects the execution of the data feed even with the the host OS, BIOS or
  any other software in the machine compromised.
As Oraclize, it relies on centralized trust sources to provide an answer.
