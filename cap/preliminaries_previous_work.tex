\section{Previous Work} \label{sec:previous_work}

There are several attempts to provide information to the blockchain from the
  outside.
By the way they gather the data we divide them in ``Distributed Oracles'', where
  the data is gathered by a group of third party participants.
Or as ``Data Feeds'', where the data is provided by a centralized party, using
  some techniques to authenticate the data.

We also add ``Trustless Distributed Casino'' as they provide a solution to a
  similar problem.
The perform trustless bet, but not  on real world events, it is limited to
  bets on random events.

\subsection{Distributed oracles}
\subsubsection{Orisi}
Orisi \cite{orisiwhitepaper} is a distributed system for bitcoin smart
  contracts that relies on multiple oracles to bring information from outside
  of the blockchain.
It allows its users to transfer money from one address to another when a
  condition is met.

In the first step both players need to chose 7 oracles, them will decide the
  bet winner.
They are chosen from ``The Oracle List'', a curated list with oracles.
A multisignature address is then generated to store the money while the bet
  takes place.
A multisignature address is defined by \texttt{m} addresses and a required
  number \texttt{n} (\texttt{n} < \texttt{m}) of them to sign. A valid
  signature for a multisignature address is generated by using at least
  \texttt{n} out of the \texttt{m} addresses defining it.

The multisignature address generated will store the money until the oracles
  decide where the transaction goes. To avoid the oracles sending the money
  to themselves the multisignature transaction include the address of the
  receiver, so we want a $1$ + ($n$ of $m$), where the extra signature is
  from the receiver. As this kind of transaction is not considered
  standard\footnote{Non standard is recognized as a valid transaction by
  everyone, however by the time this article was written only about the $5\%$ of
  the mining power will mine it. Including this transaction in the blockchain
  will take on average more time than a standard one.},
  Orisi uses a bigest multisignature address, where instead of using
  $n$ out of $m$ oracles, it adds more receiver keys. Requiring $m$ + $1$
  signatures of $2m$ - $n$ + $1$. With this configuration the oracles are
  not able to move the money by themselves, and at least one signature
  from the receiver is required.

\subsection{Trustless distributed casino}

\subsubsection{Winsome.io}
In may 2016 Rouleth \cite{winsomeio} was launched as a distributed application
  on the ethereum network. Offering its players a ``provably-fair'', real money
  roulette.
``BlockJack'' was launched in early 2017 using the Ethereum network.
It is the first playable blackjack game on the Ethereum mainnet.

Winsome.io is the instance where these games are enclosed and offers unique
  advantages over traditional casinos (physical and virtual), like no need to
  trust on it, and complete control over the funds the entire time while
  playing.
It does work in a distributed fashion using smart contracts, publicly availables
  for everyone's scrutiny.

Winsome.io provides its users trustless gambling over random events.
By using the ethereum network as backend.
It has been quite successful and one of the most popular descentralized
  applications on the Ethereum Network.

\subsection{Secured data feeds}
A secured data feed is a centralized stream of data, where a trusted entity
  ensures the correctness of the data provided by it.
It is usually protected from tampering by cryptographic algorithms.
ALthough this is a centralized data source, using many techniques or providers
  might be combined to get a decentralized source.

\subsubsection{Oraclize}
Oraclize \cite{oraclizeit} provides an interface for using data fetched from a
  web site in the ethereum blockchain, working with arbitrary URLs or queries
  in certain web services, as ``Wolfram Alpha''\footnote{Wolfram Alpha is a
  knowledge engine able to answer queries rather than provide links to data
  sources, as a search engine does.}.
It provides an Authenticity Proof of the data gathered, so the user can check
  that the data provided by the interface was generated by the source and had
  not been tampered.

\subsubsection{Town Crier}
Town Crier \cite{zhang2016town} is an authenticated data feed system for
  the ethereum blockchain.
As oraclize it works as a bridge between web feeds, and the blockchain.
It uses an Intel technology called ``Software Guard Extensions''
  \cite{costan2016intel}, than provides some execution guarantees of the software
  executed by hardware protected areas.
This protects the execution of the data feed even with the the host OS, BIOS or
  any other software in the machine compromised.
