\section{Previous Work}

There are several attempts to provide information to the blockchain from the
  outside, by the way they gather the data we divide them in
  ``Distributed Oracles'', where the data is gathered by a group of third party
  participants. Or as ``Data Feeds'', where the data is provided by a
  centralized party, using some techniques to authenticate the data.

We also add ``Trustless Distributed Casino'' as a solution to a similar problem,
  it does not provide information about the real world, but it does perform
  backed by the ethereum blockchain, any participant has entire power over its
  money at any point, however is limited to gambles on random events.

\subsection{Distributed oracles}
\subsubsection{Orisi}
Orisi\cite{orisiwhitepaper} is a distributed system for bitcoin smart
  contracts that relies in multiple oracles to bring information from outside
  of the blockchain.
It allows its users to transfer money from one address to another when a
  condition is met.

Both players agree on 7 oracles to be used to decide the transfer, usually
  chosen from ``The Oracle List'', a curated list with oracles. But could also
  be chosen from any other place the players want. Then, a multisignature
  address is generated to store the money while the bet takes place.
A multisignature address is defined by \texttt{m} addresses and a required
  number \texttt{n} (\texttt{n} < \texttt{m}) of them to sign. A valid
  signature for a multisignature address is generated by using at least
  \texttt{n} out of the \texttt{m} addresses defining it.

The multisignature address generated will store the money until the oracles
  decide where the transaction goes. To avoid the oracles sending the money
  to themselves the multisignature transaction include the address of the
  receiver, so we want a $1$ + ($n$ of $m$), where the extra signature is
  from the receiver. As this kind of transaction is not considered standard
  \footnote{Non standard is recognized as a valid transaction by everyone,
  however by the time this article was written only about the $5\%$ of the
  mining power will not process it. Including this transaction in the
  blockchain will take on average much more time than a standard one.},
  Orisi uses a bigest multisignature address, where instead of using
  $n$ out of $m$ oracles, it adds more receiver keys. Requiring $m$ + $1$
  signatures of $2m$ - $n$ + $1$. With this configuration the oracles are
  not able to move the money by themselves, and at least one signature
  from the receiver is required.

\subsection{Trustless distributed casino}

\subsubsection{Winsome.io}
In may 2016 Rouleth\cite{winsomeio} was launched as a distributed application
  on the ethereum network. Offering its players a ``provably-fair'', real money
  roulette.
Later, in early 2017, also using the ethereum network ``BlockJack'' was
  launched, the first playable blackjack game on the Ethereum mainnet.

Winsome.io is the instance where these games are enclosed, it offers unique
  advantages over traditional casinos (physical and virtual), like trusless,
  and complete control over the funds the entire time while playing. It does
  work in a distributed fashion using smart contracts, publicly availables
  for everyone's scrutiny.

Winsome.io provides its users trustless gambling over random events, by using
  the ethereum network as backend. It have been quite successful, it is one
  of the most popular descentralized applications on the Ethereum Network.

\subsection{Secured data feeds}

\subsubsection{Oraclize}
Oraclize\cite{oraclizeit} provides an interface for using data fetched from a
  web site in the ethereum blockchain, it works with arbitrary URLs or queries
  in certain web services, as ``Wolfram Alpha'' \footnote{Wolfram Alpha is a
  knowledge engine, able to answer queries rather than provide links to data
  sources, as a search engine does.}.
It provides an Authenticity Proof of the data gathered, so the user can check
  the data provided by the interface was generated by the source and have not
  been tampered.

\subsubsection{Town Crier}
Town Crier\cite{zhang2016town} is an authenticated data feed system for
  the ethereum blockchain, as oraclize it works as a bridge between web feeds,
  and the blockchain. It uses an Intel technology called ``Software Guard
  Extensions''\cite{costan2016intel}, than provide some execution guarantees of
  the software executed by hardware protected areas. This protects the execution
  of the data feed even with the the host OS, BIOS or any other piece of the
  machine compromised.
