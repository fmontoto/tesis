Bitcoin is the first fully distributed cryptocurrency made publicly available,
  it was proposed in 2008 by Satoshi Nakamoto (a pseudonym)
  \cite{nakamoto2008bitcoin}.
The same author shared as open source code a implementation of the protocol in
  January 2009. And the protocol has being running since then.

Nevertheless, Bitcoin is not the first idea of electronic cash.
The idea of electronic cash has been present within the cryptographic community
  since at least 1983, when Chaum \cite{chaum1983blind} proposed a system for
  anonymous payments.
And the attempts kept going for other three decades, hundreds of paper have
  been published with improvements of e-cash schemes\cite{barber2012bitter}.
So, why is Bitcoin so popular and achieved the notority that three decades of
  academic research on the field could not achieve?

Barber et al.\cite{barber2012bitter} suggest a few key points to explain why
  was Bitcoin the first electronic currency to take off.
\begin{enumerate}
\item No central point of trust.
	Bitcoin is a fully distributed system, there are no trusted entities in the
	  system. The only asumption is that the majority of the network participants
	  are honests. Every previous proposal had a central trusted entity for
	  critical tasks, as preventing double spending and coin issuance.
\item Predictable money supply.
	The money supply is minted at a defined and transparent rate, defined from the
	begining of the protocol.
\item Transaction irreversibility.
	Bitcoin transactions quickly become irreversible. This is a big difference with
	  credit cards, where chargebacks has been using largely to commit frauds.
\end{enumerate}
Bitcoin has not stopped to gain massive popularity and attention from the press.
Mainly because its market capitalization (over USD $36000000000$), and some
  illegal activities it has been using to as ransom to retrieve victim's data
  encrypted for malicious software, or as exchange medium in one of the most
  famous online black market, closed in 2013 by the FBI.

  The main technical advance in Bitcoin is its database, the
  \texttt{blockchain}\cite{inventionblockchain}\cite{blockchainmostimportant}.
The blockchain is a distributed database formed by an always growing list of
  blocks, where each block contains the data to be stored, a timestamp and a
  link to a previous block. Its fully distributed nature allows bitcoin to lack
  a central authority.

\subsection{Blockchain}
  The blockchain works as the bitcoin's ledger, it keeps record of all
transactions and coin generation that had ever taken place in the protocol.
It is completely distributed and public, anybody can participate in the
protocol and get a copy of it. This makes simple to prevent double spending
and be sure the received coins are valid, as anybody can examine where each
coin came from.

As any other distributed system, the blockchain must resolve the consesus
  problem \cite{fischer1983consensus}. Get all the participants to agree on
  the data. This is a fundamental problem to any distributed system. In the
  the blockchain anybody with an internet connection can be part of the
  protocol, so solving this problem is quite challenging. Some authors argue
  the blockchain is the first practical solution to the Byzantine Consensus
  problem \cite{miller2014anonymous} \cite{sun2014solving}.

The blockchain does not introduce a new primitive or idea to solve all of the
  mentioned problems, it does use existing tools in an innovative way to.
To keep a consistent history the blocks are linked using the hash of the
  previous block, this prevents tamper on past data.
\begin{figure}
	\centering
	\def\svgwidth{\columnwidth}
	\includesvg{block_links}
	\caption{Blocks linked to each other in the blockchain}
	\label{fig:block_links}
\end{figure}



As it is public, every transaction can be verified by any protocol par

  The blockchain is  a chain of blocks were each of them keeps a link
to the previou
  also fully distributed. In bitcoin


