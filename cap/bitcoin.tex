\section{Bitcoin}
Bitcoin is the first fully distributed cryptocurrency made publicly available,
  it was proposed in 2008 by Satoshi Nakamoto (a pseudonym)
  \cite{nakamoto2008bitcoin}.
The same author shared as open source code a implementation of the protocol in
  January 2009. And the protocol has being running since then.

Nevertheless, Bitcoin is not the first idea of electronic cash.
The idea of electronic cash has been present within the cryptographic community
  since at least 1983, when Chaum \cite{chaum1983blind} proposed a system for
  anonymous payments.
And the attempts kept going for other three decades, hundreds of paper have
  been published with improvements of e-cash schemes\cite{barber2012bitter}.
So, why is Bitcoin so popular and achieved the notority that three decades of
  academic research on the field could not achieve?

Barber et al.\cite{barber2012bitter} suggest a few key points to explain why
  was Bitcoin the first electronic currency to take off.
\begin{enumerate}
\item No central point of trust.
	Bitcoin is a fully distributed system, there are no trusted entities in the
	  system. The only asumption is that the majority of the network participants
	  are honests. Every previous proposal had a central trusted entity for
	  critical tasks, as preventing double spending and coin issuance.
\item Predictable money supply.
	The money supply is minted at a defined and transparent rate, defined from the
	begining of the protocol.
\item Transaction irreversibility.
	Bitcoin transactions quickly become irreversible. This is a big difference with
	  credit cards, where chargebacks has been using largely to commit frauds.
\end{enumerate}
Bitcoin has not stopped to gain massive popularity and attention from the press.
Mainly because its market capitalization (over USD $36000000000$), and some
  illegal activities it has been using to as ransom to retrieve victim's data
  encrypted for malicious software, or as exchange medium in one of the most
  famous online black market, closed in 2013 by the FBI.

  The main technical advance in Bitcoin is its database, the
  \texttt{blockchain}\cite{inventionblockchain}\cite{blockchainmostimportant}.
The blockchain is a distributed database formed by an always growing list of
  blocks, where each block contains the data to be stored, a timestamp and a
  link to a previous block. Its fully distributed nature allows bitcoin to lack
  a central authority.

\subsection{Blockchain}
It works as the bitcoin's ledger, it keeps record of all
  transactions and coin generation that had ever taken place in the protocol.
It is completely distributed and public, anybody can participate in the
  protocol and get a copy of it. This makes simple to prevent double spending
  and be sure the received coins are valid, as anybody can examine where each
  coin came from.

As any other distributed system, the blockchain must resolve the consesus
  problem \cite{fischer1983consensus}. Get all the participants to agree on
  the data. This is a fundamental problem to any distributed system. In the
  the blockchain anybody with an internet connection can be part of the
  protocol, so solving this problem is quite challenging. Some authors argue
  the blockchain is the first practical solution to the Byzantine Consensus
  problem \cite{miller2014anonymous} \cite{sun2014solving}.

Proof of work is the algorithm used by the bitcoin blockchain to seek
  consensus. Each entity trying to add data to the database must proof it
  has done some required work. This algotithms was designed originally to
  fight the email spam, by requiring the sender of an email to prove a small
  work was done in order to send the email\cite{dwork1992pricing}. This is
  achieved by using a hard to calculate, but easy to check function. This
  way the receiver or the mail server can easily check if the sender did
  the required work, however this work was much harder. The difficulty of
  a work is defined by the amount of computational power required to get
  it done.

The atomical piece in the blockchain is the block. Each valid block carries with
  itself a proof of work, so every entity trying to get a valid block into the
  database tries to solve a puzzle to get this proof.
This process is called mining, therefore the entities trying to get a valid
  block are called miners.
By design a block must be produced every 10 minutes, so the work required to
  mine a block is a ajusted periodically to met this goal.

The proof of work consists in building a block with a hash under a threshold
  value, so the miners should reorder and change the block until the hash
  fulfill the requirement. There is not a known algorithm to do this in a
  better way than brute force, so the only method to get a hash that mets the
  criteria is to try with different block configurations, there are some bytes
  of nonce, a timestamp and transactions to be changes to get different hashes.

The structure of a Bitcoin block is show In the figure \ref{fig:block_pow}, the
  fields with the light purple background represents the block header, the data
  hashed to get the block's hash. The transactions are indirectly hashed in the
  Merkle Root\footnote{A \texttt{Merkle Tree} is a tree in which each non leaf
  node is labeled with the hash of its children's labels. In the block each
  transaction is mapped into tree's leaf. So the root of this tree hashes all
  the transactions}.

\begin{figure}
	\centering
	\includesvg[width=\columnwidth]{block_pow}
	\caption{Block structure}
	\label{fig:block_pow}
\end{figure}

  characterisitic
  A block holds the basic operations in the protocol, the transactions. The
  proof of wo
A block is a

The blockchain does not introduce a new primitive or idea to solve all of the
  mentioned problems, it does use existing tools in an innovative way to.
To keep a consistent history the blocks are linked using the hash of the
  previous block, this prevents tamper on past data.
\begin{figure}
	\centering
	\def\svgwidth{\columnwidth}
	\includesvg{block_links}
	\caption{Blocks linked to each other in the blockchain}
	\label{fig:block_links}
\end{figure}



As it is public, every transaction can be verified by any protocol participant.


  The blockchain is  a chain of blocks were each of them keeps a link
to the previou
  also fully distributed. In bitcoin


