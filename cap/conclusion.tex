Since the creation of Bitcoin in 2008, more than a thousands of different
  cryptocurrencies have been proposed, most of them following Bitcoin's main
  design features.
Bitcoin is known and often criticized due to its slow evolution and resistance
  to changes.
This is further stressed by the overwhelming number of different and fast
  evolving currencies created.

A currency where rules does not change too much over time makes it a good store
  for value, and likely this Bitcoin's property contributes to make it the
  most valuable cryptocurrency today, as measured by market capitalization.
This is a desirable feature for our algorithm.
We do not want the value of the money when the bet is placed to be too different
  to the one when the bet
  is resolved.

However, using Bitcoin in our protocol does not come for free.
This slow change rate keeps bitcoin script language very limited.
When people talk about smart contract or scripting, Bitcoin is never the first
  currency coming to people's mind.
The limited scripting, slow change rate and the biggest network made Bitcoin our
  choice:
The script limitation poses a real challenge in the protocol presented, making
  this an open problem.
As we discussed at Section~\ref{sec:discussion_previous_work}, the problem has
  many implemented solutions for currencies with more expressive scripting
  languages.
Bitcoin's slow change rate helps to keep the same rules from the start to the
  end of a bet, thus making the platform suitable for long term bets.
Finally, having the largest network enforcing the protocol rules makes more
  difficult some known attacks (as \emph{Majority and Eclipse
  \cite{heilman2015eclipse}}) where attackers can overtake the control of the
  network or part of it.
Another cost of using a limited scripting language is the amount of instructions
  and transactions required to express the protocol, which translates into
  somewhat higher costs.

The cost of running our proposed algorithm may be prohibitively expensive for
  some cases.
Nonetheless we believe there are people willing to afford it when doing betting
  on large sums of money, as the cost of the protocol would represent a small
  fraction of the bet.
Notice that the cost does not increase with the amount of money involved.

Most of the transactions required by the protocol are called
  \emph{``Non standard''} in Bitcoin.
Non standard transactions are valid and everyone recognizes them as such.
Yet, most of the miners will not mine them, so submitting the transactions
  will take more time than a standard one.
We hope new proposals that makes use of this available features, like this one,
  can move the Bitcoin community to embrace this ``non standard'' transactions.

One of the most challenging problems we had to address for this protocol was to
  translate the protocol restrictions to Bitcoin transactions.
Studied existing solutions are composed just by one or two different
  transactions.
In Ethereum one smart contract is enough for the applications we discussed at
  Section~\ref{sec:previous_work}.
For Orisi, only one transaction is post in the blockchain with the bet, all
  the other communication and required set up happens outside the blockchain.
Our protocol uses the blockchain widely, not only for money transfers, but also
  for oracle discovery and initial communication.
This seeks to avoid extra dependencies and keep most of the protocol expressed
  in the same framework.
This come to a cost, more and larger transactions.
They are not free anymore: we are charged by the miner for each byte we post in
  the blockchain in a transaction.
This cost is discussed at Section~\ref{sec:cost_analysis}.

A good example of this complexity is the transaction shown in
  Section~\ref{app:winner_collect}.
Bitcoin provides a multi-sig check function, which verifies if a certain number
  of signatures under the provided public keys are present.
In the Bet transaction we need to implement our own version of this
  cryptographic primitive, in order to get a multi-hash check function, which
  checks that at least a threshold number of pre images are provided for the
  written hash.
This transaction is hard to write as it can not be written using loops (because
  the language does not support them), and even not used branches need to be
  included in the transaction to match the original hash in the output of the
  Bet transaction.

Our initial decision of actually implementing the protocol proved to be very
  helpful when evaluating limitations of the protocol.
It allowed us to calculate the size of the required transactions, which were
  then used to get a very accurate cost of the fees per transaction.
Another limitation found thanks to the developing of an implementation was the
  existence of a maximum size of an element on the Bitcoin stack: 520 bytes.
Due this limitation our original implementation was able to use up to 6 oracles,
  after some script optimizations our implementation supported up to 7 oracles.
With other optimizations we might be able to support one or two more oracles.
However, without a change in the way Pay To Script Hash transactions work, or
  the hard limit of 520 bytes per element in the stack, it is impossible for our
  protocol to support a larger number of oracles.

Much of the work in the implementation involved basic tools to manipulate and
  generate custom transactions.
This code is a contribution not only for our protocol, but also for people
  willing to use the bitcoin with custom transactions, as we do.

Gambling is a really big and heavily regulated industry, initiatives like
  Winsomeio, oraclizeit, or this protocol are tools to hopefully  make this
  industry more flexible.
They aim to allow people to fully control its money while gambling, contrarily
  from standard casinos and betting sites.
As cryptocurrencies allows people to stay away from a centralized government
  issue currencies, this kind of tools also allows people to move away from
  centralized gambling sites.
There is yet another big difference.
People running protocols on top of blockchains can see every line of code
  executed and how decisions are taken, making gambling a transparent process,
  opposed as it is today, where people uses only the public interface of a web
  site or brick and mortar casino and what it is ``behind the scenes'' remains
  opaque.
Never seeing what is really going on ``behind the scenes''.

We believe this work is a contribution for both worlds: It introduces new
  options in the gambling industry. And it also adds a new application on top of
  Bitcoin, using it as more than a value store currency.
Bitcoin introduces a method to store and transfer value without relying in a
  central authority.
The presented protocol uses this property and introduces an option to do
  these transfers subject to real world events, without sacrificing the lack
  of central control.
Therefore we see the objective for this work as accomplished.

While working on this project we realized bets are only one example of external
  events that may decide the destination of somebody else's money.
We could also see our oracles as ``judges'' in a dispute.
And instead of providing an event outcome into the blockchain, they can
  decide on any subject, acting as arbitrator on a controversy.
Small adjustments would be required for the protocol.
For example, if we remove the random selection of oracles, probably judges need
  to be agreed beforehand by the players.
Other required change is to remove the penalization by answering different
  from the majority.
It is perfectly fine and even desirable to have different opinions among the
  judges, and the ones with the minority should not be punished.

It might be also desirable to have a larger number of oracles participating in
  a bet.
However, this requires a change in the Bitcoin implementation, which is beyond
  under our control.

Future work might also include sharing oracles among different bets.
This could help making the protocol cheaper but adds new issues, such as paying
  the oracle.
Our proposal pays the oracles in the same Bet transaction, get all the users to
  finance the oracles would require a new approach.

Other interesting change would be to decrease the number of transactions or
  even the transactions' size to decrease the fees.
There are new features being introduced in Bitcoin which may help on this
  regard.
For example, one of the last additions to Bitcoin was ``Segregated Witness''
  (to solve transaction malleability), which opens the door to efficient and
  practical off the blockchain transactions.
This is an option to decrease the number of required transactions posted in the
  blockchain, and therefore decrease the amount of fees paid.
