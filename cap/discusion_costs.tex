\section{Costs}
At section \ref{sec:cost_analysis} the cost of a successful transaction is
  provided, it sums up to over USD 60.
A smaller number of oracles or less payment could bring down this number,
  however even after this the number will not be negligible.
This has some important consequences for this protocol, its expensive to run,
  and unsuitable for small amount bets.
Players might be willing to pay to run the protocol, but unlikely for small
  bets.

\subsection{Individual Attacks}

\subsubsection{Oracle} \label{subsec:individual_attack_oracle}
Within the protocol, oracles maximize they earnings by participating and
  voting for winner.
However, if oracles stop participating after they send the
  \textit{Oracle Inscription} and $c/n > \fee{Oracle Inscription}$ they can
  also win money:

\begin{center}
    \begin{tabular}{|c|c|}
        \hline
          \textbf{Item} & Cost[USD] \\
        \hline
          Initial deposit & -[Oracle Payment] - [Two Answers Penalty] \\
        \hline
          Redeem initial deposit & [Oracle Payment] + c/n - \fee{Oracle Inscription} \\
        \hline
          Redeem initial deposit fee & - \fee{Redeem Initial Deposit} \\
        \hline
          Redeem two answers penalty & [Two Answers Penalty] \\
        \hline
          Redeem two answers penalty fee & - \fee{Redeem Two Answers Penalty} \\
        \hline
          \textbf{Total} & c/n - \fee{OracleInscription} - \fee{RedeemTwoAnswersPenalty} \\
        \hline
    \end{tabular}
    \captionof{table}{Oracle exits after Oracle Inscription}
    \label{tab:oracle_abort}
\end{center}

If c/n is bigger than the fees required to redeem the
  \textit{Oracle Inscription} transaction, oracle wins money by aborting after
  submitting this transaction.

For the players this bring other cost beside the c/n amount.
They need to get a replacement for this oracle, a new smaller
  \textit{Bet Promise} transaction can be sent.
Which has an added cost, as it have to pay fees.

A malicious oracle can try to attack the players by aborting after the
  \textit{Oracle Inscription} transaction, resulting in an extra cost to the
  players.
Decreasing the value of c makes this attack unlikely, as oracles will require
  to spend money in order to perform it.
However decreasing this value might discourage oracles to participate in the
  bet.
A second \textit{Bet Promise} transaction to ask for another oracle is much
  cheaper than the original one, as it can link the bet data from to the first
  one, and only requires to list one oracle.
So players can spend some extra money trying to get a new oracle, run the bet
  with less oracles or abort it.
Running the bet with fewer oracles does is the only option that does not cost
  extra money.

This attack is limited, a malicious oracle can trigger the attack only once, as
  it will replaced or ignored after doing it.
But if the same bet get multiple malicious oracles they can delay the bet long
  enough to forve player to abort it.
This is extremely unlikely if oracles are being selected randomly from a big
  pool.

\subsubsection{Player} \label{subsec:individual_attack_player}
If the bet faces a malicious player, it can abort the execution at any point
  before submitting the \textit{Bet} transaction.
As during the protocol both player contribute in the same way to pay it, if it
  gets cancelled at any moment, both players get to lose the same amount of
  money.
So is feasible for a malicious player to attack the other one, but the cost
  of the attack is the same of the harm done.
Other important measure the protocol has to limit the extent of this attack
  is to keep the bet money separated from the fees and oracle payments.
A malicious player can not make the other one to lose more than the bet
  associated costs.

\subsubsection{Colluded attacks}\label{subsec:colluded_attacks}

For every collusion involving the player and any number of oracles less than
  the majority threshold, the damage is limited for the bet associated costs,
  and they behave as the individual attacks described in sections
  \ref{subsec:individual_attack_oracle} and
  \ref{subsec:individual_attack_player}.
The cost of any of this attacks is the same of bigger than the damage done,
  so we say they are infeasible.

When one player and at least the majority threshold of the oracles are colluded,
  this player can take all the money to distribute it with the colluded oracles.
This cost to the honest player all of the bet money plus the associated costs.
This attack is discussed at section \ref{subsec:inc_oracles}.
