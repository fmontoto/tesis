\section{Digital Signatures}
The idea of \textit{digital signature} was introduced in 1976 by Diffie and
  Hellman in \textit{New Directions in Cryptography}''\cite{diffie1976new}. This work
  introduced what they called \textit{Public Key Cryptosystem}, where enciphering and
  deciphering operations use different keys, $E$ and $D$, such that computing D
  from E is computationally infeasible.
Today this pair is widely used and known as Public Key (PK) and Secret Key (SK).

The public key cryptosystem, or asymmetrical cryptography was created to solve
  one important problem of symmetrical systems\footnote{As opposed to
  asymmetrical systems, symmetrical systems use the same key to cipher and decipher
  the messages.}: It is impossible to start a secure communication in an
  insecure channel without previously exchange of a key using a secure channel.
To establish secure communication within an insecure channel participants makes
  its PK publicly available to the others.
Anyone willing to talk to another participant must cipher its message using the
  public key of the receiver, this way the only one able to decipher the message
  is the intended receiver.

A digital signature, as its name indicates, is a mechanism to provide protection
  against third party forgeries. It must be easy for anyone to evaluate if the signature is
  authentic, but impossible for anyone but the signer to produce it. This is
  especially challenging since any digital signal can be easily copied.

The signer uses his SK to produce a signature over the message to sign, and
  anyone with the signer PK and the message can determine the validity of the
  signature.

The most important property of a digital signature is that it does not matter how
  many pairs \mbox{\textless message, signature\textgreater} a third party has
  seen, it does not make it easier to generate a signature for a new
  message. This property is called existential unforgeability~\cite{dwork1994efficient}
  and it is formally defined as follows:

A signature scheme is existentially unforgeable if, given any polynomial (in the
  security parameter) number of pairs
$$ (m_{1}, S(m_{1})), (m_{2}, S(m_{2})),\cdots,(m_{k}, S(m_{k})) $$

where $S(m)$ denotes the signature on the message $m$, it is computationally
  infeasible to generate a pair $(m_{k + 1}, S(m_{k+1}))$ for any message
  $m_{k+1} \notin \{m_{1},\cdots,m_{k}\}$.

Sometimes we want signature schemes where more than one signer is needed to produce a signature.
We call them multisignature schemes~\cite{itakura1983public}.
In the general case a multisignature scheme has two associated parameters: The number
  of signers able to sign; and the required numbers of signers to be a valid
  signature.
A multisignature with 7 signers where at least 4 of them are required to produce
  a valid signature is called a 4 of 7 multisignature.
So, a signature is valid with at least 4 out of the 7 valid signers.
