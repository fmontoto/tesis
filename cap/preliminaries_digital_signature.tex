\section{Digital Signatures}
The idea of ``Digital Signature'' was introduced in 1976 by Diffie and
  Hellman in ``New Directions in Cryptography''\cite{diffie1976new}. This work
  introduced what they called ``Public Key Cryptosystem'', where enciphering and
  deciphering operations use different keys, $E$ and $D$, such that computing D
  from E is computationally infeasible.
Today this pair is widely used and known as Public Key (PK) and Secret Key (SK).

The public key cryptosystem, or asymmetrical cryptography was created to solve
  one important problem of symmetrical systems\footnote{As opposed to the
  asymmetrical one, symmetrical systems use the same key to cipher and decipher
  the messages.}: It is imposible to start a secured communication in an
  insecure channel without previously exhange of a key using a secure channel.
To stablish secure communication within an insecure channel participants makes
  its PK publicly available to the others.
Anyone willing to talk to another participant must cipher its message using the
  public key of the receiver, this way the only one able to decipher the message
  is the intended receiver.

A digital signature, as its name indicates, is a mechanism to provide protection
  against third party forgeries. It must be easy for anyone to recognize as
  authentic, but impossible for anyone but the signer to produce it. This is
  especially challenging since any digital signal can be easily copied.

The signer uses his SK to produce a signature over the message to sign, and
  anyone with the signer PK and the message can determine the validity of the
  signature.

The most important property of a digital signature is that it does not matter how
  many pairs \mbox{\textless message, signature\textgreater} a third party has
  seen, but it does not make it easier to generate a signature for a new
  message.

Sometimes we require more than one signer to produce a signature, when this is
  the case we call it multisignature.
In the general case a multisignature has two associated parameters: The number
  of signers ables to sign; and the required numbers of signers to be a valid
  signature.
A multisignature with 7 signers were at least 4 of them are required to produce
  a valid signature is called a 4 of 7 multisignature.
So, a signature is valid with at least 4 out of the 7 valid signers.
