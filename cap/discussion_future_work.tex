\section{Future work}
\subsection{Protocol extension}

In our proposed protocol there is nothing that binds the oracle answer with an
  event's outcome.
We could ask the oracles an opinion, to take a position in a dispute, or
  anything we want without changing the protocol.
In a more general view, our protocol allows a pair of peers to move money
  between two options outsorcing the decision.
A bet is a particular case.

Even though our protocol is inspired by gambling, we strongly believe there is
  room to solve many other scenarios.
Lawsuits  seeking compensations is another example of deciding between two
  destinations.
Accused must pay or keep its money, and the decision is made by third parties.
Disputes on agreements and products warranties are other similar schemas.

In general we think, oracles can be repurposed as ``judges'' and let them
  decide on other scenarios.
An arbiter or court might have problems to enforce its orders, using our
  protocol the money's recipient can get it as soon as the judges agree.

\subsection{Oracle reutilization}
Utilizing oracles for more than one Bet could be another interesting addition to
  the proposed protocol.
First, because it would help to bring the cost of the protocol down.
If the oracles payments can be shared among many bets, each bet pays
  an smaller amount.
And second, if an oracle participates in many bets, it would receive a bigger
  payment than when participating on a single one.
This makes more expensive to bribe an oracle by a single player.
However it also adds a new attack possibility, multiple players betting on the
  same output could share the bribe cost.
Which is more complicated than the single player case, because requires
  coordination and new transactions among the players.

Utilizing the same answers for many bets requires changes on how the oracle get
  paid and communication between players at different bets.
It would require important changes in the current transactions.
