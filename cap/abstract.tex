Every time we place a bet, we implicitly trust the other participant  to pay in
  the case we win.
For this reason if we deal with people we do not know or trust, we do not bet
  directly with them, and instead we use a casino or other gambling sites.
We expect the law or even casino reputation to help us to enforce good
  behavior.\\
\noindent
Cryptocurrencies made possible to store and transfer value easily without
  using centralized control.
Given its recent popularity, many decentralized applications have been proposed
  and implemented on top of them.
Among them, gambling.
It is currently possible to safely bet with an stranger on a random events
  using cryptocurrencies.
We can bet on the value of digital dice, cards or roulette.
Neither trust nor centralized entities are required.
In this work we propose a protocol on top of Bitcoin to place bets on real world
  events in a decentralized and trustless fashion.
\noindent
Our protocol uses a set ``oracles'' to bring the outcome of the event on which
  the bet depends into the blockchain.
When a certain threshold of the oracles have reported the same outcome for
  the event, we unlock the winner's prize.
The protocol was designed to automatically move the money to one of the players
  in that case, so oracles do not take control of the prize, they only decide
  between two possible outcomes.\\
\noindent
Economic incentives are placed to encourage good behavior on the participants.
Oracles get paid if and only if they provide the right answer, where right is
  defined as the answer provided by the majority of them.
Also, oracles are required to place a deposit to participate.
This allows the protocol to establish economic penalties if the oracle fails to
  answer.\\
\noindent
Once the bet is placed, the prize is locked waiting for the oracles' answer.
Players relinquish their control over the money until the oracles decide the
  winner or a previously defined timeout expires.
In case of timeout, each player gets half of the prize.\\
\noindent
Using Bitcoin scripting language, the protocol guarantees fairness and execution
  correctness with both, a dishonest minority of oracles as well as a dishonest
  peer, at least once the bet is placed.
Before that a dishonest peer can cause limited economic damage to an honest
  player.
Yet the attack produces no net gain: it costs the attacker as much as the amount
  of the damage done to the victim.
Dishonest oracles, before the bet is placed, can cause small economic damage to
  both players equally, although the cost for this attack is a parameter set by
  the players.\\
\noindent
The protocol was implemented and run in the Bitcoin testnet using the official
  client to verify the transactions.
Our implementation is currently available as an Open Source project.
Discussion on the trade off between the protocol cost and economic incentives
  for the participants are included in this work.
We also provide a detail cost analysis of running the protocol.\\
\noindent
We believe this work's contribution is a novel way of reasoning and deciding
  about ``real world'' events while integrating the outcome into the blockchain.
The protocol can be easily extended to resolve contract disputes and similar
  situations where subjective but fair resolutions are needed, all in a
  decentralized environment.
