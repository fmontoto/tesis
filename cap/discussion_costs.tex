\section{Costs}
In Section~\ref{sec:cost_analysis} we discussed the cost of a successful
  transaction, which summed up to over USD 60.
A smaller number of oracles or less payment could bring down this number,
  however even after this the number will not be negligible.
This has some important consequences for this protocol: it is expensive to run,
  and that makes it unsuitable for small amount bets.

\subsection{Attacks}

\subsubsection{Oracle} \label{subsec:individual_attack_oracle}
Within the protocol, oracles maximize they earnings by participating and
  selecting ``the real winner''.
However, if oracles stop participating after they send the
  \textit{Oracle Enrollment} and $c/n > \fee{Oracle Enrollment}$ they can
  also win money:

\begin{center}
    \begin{tabular}{|c|c|}
        \hline
          \textbf{Item} & \textbf{Cost [USD]} \\
        \hline
          Initial deposit & -[Oracle Payment] - [Two Answers Penalty] \\
        \hline
          Redeem initial deposit & [Oracle Payment] + c/n - \fee{Oracle Enrollment} \\
        \hline
          Redeem initial deposit fee & - \fee{Redeem Initial Deposit} \\
        \hline
          Redeem two answers penalty & [Two Answers Penalty] \\
        \hline
          Redeem two answers penalty fee & - \fee{Redeem Two Answers Penalty} \\
        \hline
          \textbf{Total} & c/n - \fee{OracleEnrollment} - \fee{Redeem Two Answers Penalty} \\
        \hline
    \end{tabular}
    \captionof{table}{Oracle exits after Oracle Enrollment}
    \label{tab:oracle_abort}
\end{center}

If $c/n$ is bigger than the fees required to redeem the
  \textit{Oracle Enrollment} transaction, oracle wins money by aborting after
  submitting this transaction.

For the players, this brings some additional costs beside $c/n$.
If the players need to get a replacement for some oracle, a new smaller
  \textit{Bet Promise} transaction can be sent.
This new transaction has an added cost, as it have to pay fees.

A malicious oracle can try to attack the players by aborting after the
  \textit{Oracle Enrollment} transaction, resulting in an extra cost to the
  players.
Decreasing the value of c makes this attack unlikely, as oracles will require
  to spend money in order to perform it.
However decreasing this value might discourage oracles to participate in the
  bet.
A second \textit{Bet Promise} transaction to ask for another oracle is much
  cheaper than the original one, as it can link the bet data from to the first
  one, and only requires to list one oracle.
So players can spend some extra money trying to get a new oracle, run the bet
  with less oracles or abort it.
Running the bet with fewer oracles does is the only option that does not cost
  extra money.

This attack is nonetheless limited.
A malicious oracle can trigger the attack only once, as it will be replaced or
  ignored after doing it.
But if the same bet get multiple malicious oracles they can delay the bet long
  enough to force player to abort it.
This is extremely unlikely if oracles are being selected randomly from a big
  pool.
Indeed, any malicious entity who wishes to control a set of oracles will have to spend an amount
  proportional to the number of controlled oracles, so it is safe to assume that
  given a large universe $U$ of $M = |U|$ oracles, the fraction $\epsilon$ of malicious oracles will be
  small, that is, $\epsilon << 1$.
Assuming players select their oracles at random,
  any set $R\subseteq U$ of oracles of constant size $r=|R|$ will have on average $r^*=r\epsilon$ malicious
  oracles.
The process of iteratively selecting (honest) oracles from a pool is a
  Bernoulli process in which we aim to obtain $r$ honest oracles where the probability of
  selecting an honest oracle at random is $1-\epsilon$.
For the protocol we suggest $r=7$ oracles, so given a pessimistic value of $\epsilon = 0.1$,
  the probability $p$ of obtaining $r=7$ honest oracles after selecting $m=12$ oracles uniformly
  at random can be modeled by the Pascal (Negative Binomial) distribution, which for the given values
  gives us a probability $p>0.997$.


\subsubsection{Players} \label{subsec:individual_attack_player}
If the bet faces a malicious player, it can abort the execution at any point
  before submitting the \textit{Bet} transaction.
As during the protocol both players contribute in the same way to pay it, if the bet
  gets canceled at any moment, both players get to lose the same amount of
  money.
So it is feasible for a malicious player to attack the other one, but the cost
  of the attack is the same of the harm done.
Other important measure the protocol has to limit the extent of this attack
  is to keep the bet money separated from the fees and oracle payments.
A malicious player can not make the other one to lose more than the bet
  associated costs.

\subsubsection{Collusion attacks}\label{subsec:colluded_attacks}

For any collusion involving the player and an arbitrary number of oracles less than
  the majority threshold, the damage is limited to losing an amount equal to the
  associated costs, discussed in
  Sections~\ref{subsec:individual_attack_oracle}~and~\ref{subsec:individual_attack_player}.
The cost of any of this attacks is of the same than the damage done, so we claim
  they are impractical.

When one player and at least the majority threshold of the oracles are colluded,
  this player can take all the money to distribute it with the colluded oracles.
This cost to the honest player all of the bet money plus the associated costs.

From table~\ref{tab:oracle_costs}, we know that the expected earning for a
  well-behaved oracle is
  [Oracle Payment] - [Registration] - $\mathcal{F}\, \cdot$ (1011 + $n \cdot 62$).
  We call this value $\mathcal{E}$.

When an oracle takes a bribe there are two possible outcomes for him in the
  protocol:
  either enough oracles also take the bribe in which case it gets the payment for a correct
  protocol execution $\mathcal{E}$; or not enough
  players take the bribe and the oracle is penalized for providing a wrong
  answer. In this last case the oracle receives
  $\mathcal{E}$ - [Oracle Payment] =
  - [Registration] - $\mathcal{F}$\, $\cdot$ (1011 + $n \cdot 62$).

A honest oracle gets $\mathcal{E}$ for a correct execution.
Any bribe $\mathcal{B}$ should be such that, at the end of the day, the
  corrupted oracle gets at least $\mathcal{E}$ in
  order to provide a sufficiently strong incentives for lying:
\begin{align}
  \mathcal{B} - [Registration] - \mathcal{F}\, \cdot (1011 + n \cdot 62) \geq \mathcal{E} \nonumber \\
  \mathcal{B} \geq [Oracle Payment]
\end{align}
Anyone willing to bribe the oracle should at least offer [Oracle Payment] as bribe.
This give us a lower bound for the cost of bribing enough oracles to change the
  outcome of a bet:
  \begin{equation}
   (\lfloor \frac{n}{2} \rfloor + 1) \cdot [Oracle Payment]
  \end{equation}
If oracles were chosen randomly, it is unlikely that the dishonest player and
  the oracles trust each other. In consequence, in a realistic attack, the final cost
  of the bribe must include the cost of all transactions required to enforce the desired
  operation of the corruped oracles.
