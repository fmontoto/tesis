\section{Gambling} \label{gambling}
Gambling is the activity of predicting events and placing a wager on the
  uncertain outcome of those events, with the intent of winning money or
  valuable goods.
A wager can be put on many different events, in a casino we find randomizing
  devices as dices, roulette wheels, etc. which are used to get randomize
  events. In other establishments we can bet on sport events, such as a horse
  racing, football games, etc. or the minimum temperature in Santiago during
  this night.
Its popularity and the big amounts of money at stake
  inevitably entails a lot of interest on this activities. Most of the time
  gambling is heavily regulated and taxed, also it is usual that lotteries are
  owned by the state.

Internet has been making cheaper to open and operate a casino, even without
  complying laws from any country.  This and the massive internet use, has been
  moving the gambling industry online\cite{shelat2002makes}
  \cite{griffiths2008internet}. The global Internet gambling market was
  estimated to be worth US\$28.32 billion in 2012 and forecasted to rise to
  US\$49.64 billion by 2017\cite{gainsbury2015risky}.
However, gambling not only takes place in casinos, lotteries or betting sites,
  it can also involve two or more individuals with no intermediaries. In Chile
  friends usually bet on their favorites football teams.

Nonetheless all the different ways for placing a bet, all of the mentioned
  share a common obstacle, participants are required to trust in the other
  parties to pay if they lose.
  Even if the bet takes place in a physical casino, where the law can enforce
  the bet, is not certain the casino will be able to pay after the resolution.
We might not be aware of the fact, but every time we place a bet we are
  implicitly trusting in a third party, either the other player or the bet
  site. For physical casinos this is usually not a problem, as they are
  regulated by the law, any misconduct can get the casino to the justice and
  even get its license revoked. As there is a significant cost on starting a
  physical casino, them are also encouraged to keep a good reputation, in order
  to get customers.

Friends usually are trusted people, so trusting them when gambling might not
  be considered an issue. Also, probably the friendship is at risk if the bet
  is not paid. Other option is to get a third friend to get the money until the
  bet result.
Online casinos on the other hand are more problematic, there are many knowns
  scam schemes, as described by Griffiths\cite{griffiths2010crime}. And
  half of the players at this sites believe the providers are cheating on
  them\cite{mcmullan2010online}. However, some of them are subject of
  government regulation and many have being in the business for several years,
  this kind of characteristics could help to indicate an online site is
  trustworthy.

But, what if you would like to gamble in a event that no gambling site offers
  nor any friend want to? Probably the internet would be the place to look for
  somebody willing to gamble on this event. Yet, how could you trust the
  potential person in order to bet with him?

\section{Cryptocurrency}
Digital currency refers to any currency stored and transferred
  electronically.
  A subset of the digital currencies is called virtual currencies: them are
  usually defined\cite{bcentraleuro} as a \itquote{ unregulated,
  digital money, which is issued and usually controlled by its developers, and
  used and accepted among the members of a specific virtual community}.

Based on the interaction of the currency with currencies outside the
  community there are three types of virtual currencies: The ones with almost
  no interaction with the outside money, this is usually the case of video
  games, where its currency is only valuable within the game. A second
  type is where the currency can be purchased directly using other currency.
  Here, we observe an unidirectional flow. The third type is when the flow is
  bi directional, the users can sell and buy the currency.
A cryptocurrency is a bi directional virtual currency, that uses cryptography
  for security and anti-counterfeiting measures. Virtual currencies are been
  historically linked to cryptography, the first known investigations
  \cite{chaum1983blind} to establish a virtual currency where lead by David
  Chaum, an American cryptographer. However, despite his and others effort
  (e-gold\footnote{https://www.wired.com/2009/06/e-gold/},
  Ecash\cite{chaum1990untraceable},
  DigiCash, LibertyReserve, among others), virtual currencies never where
  massively adopted.

By late 2008, using a pseudonym, was released a short
  whitepaper\cite{nakamoto2008bitcoin} with yet another virtual currency
  protocol specification. A few months later, during 2009 its implementation
  was made available as open source code. The main difference with previous
  implementations was its lack of a central organization, this new coin was
  completely decentralized. The software started to being run by some early
  enthusiasts and Bitcoin gave the step from an idea to an usable coin. The
  first years was the coins were exchanged for free among the community users.
  However, at some point the community was big enough and its members started
  to give value to the coin, then the first exchanges from and to other coins
  started to take place. Bitcoin transitioned into a bi directional flow
  virtual coin.

Then the first online exchanges between bitcoin and other currencies started
  to appear, the coin started to gain traction as people outside the community
  were able to buy and sell coins. As the money became popular, the idea was
  taken and a whole generation of cryptocurrencies were born. Today the
  market capitalization of Bitcoin (this is, the amount of money times its
  value in USD) is over 25,000,000,000 USD.

\section{Gambling using Cryptocurrencies}

With cryptocurrencies getting more and more popular, it was only a matter of
  time until the first sites started to offer started to offer some games of
  chance an act as online casinos. Where the only difference with a traditional
  online casino was the currency on which the bet takes place.
However, as any other currency online casino, any player who decided to play
  here is at the mercy of the casino. If the casino does not want or does not
  have the means to pay, there is nothing the participant can do and its money
  is lose. More on online casinos at subsection \ref{gambling}. The problems
  described for online casinos using traditional currencies apply in the same
  way to the new ones.

After some time, people started to see some potential on cryptocurrencies to
  solve some of the trust issues related to gamble. In 2014 Andrychowicz et
  al.\citation{andrychowicz2014secure} proposed a two party randomized gambling
  protocol. Players are not required to trust each other in order to gamble, so
  even if the loser does not behave correctly the honest player, can get its
  prize.
The protocol is not a representation of a casino game, but effectively allows
  player to gamble on a random event.
Also in 2014, a group of Bitcoin enthusiasts started
  Orisi\footnote{http://orisi.org},  a distributed oracles system for
  cryptocurrency contracts.
Orisi allows users to access data of the outside world from the blockchain,
  by using a distributed set of oracles. So instead of trusting in one instance
  to provide the data, the trust is placed in the majority of several different
  oracles.
More recently, on early 2017, Winsome\footnote{https://www.winsome.io} was
  released. Advertised as a \itquote{Provably Fair | Trustless Casino}, Winsome
  is an online casino where wager are placed in a public smart contract posted
  in the Ethereum's blockchain. So the contract, defining the game, is enforced
  by the Ethereum protocol. As May 2017, they do offer two casino games,
  blackjack and \texttt{Rouleth}, an online roulette.

Motivated to provide an option to gamble over real world events with untrusted
 peers. This work proposes a protocol to define the destination of an initial
 wage between the two player. The decision is taken by a set of oracles, which
 are being paid also within the protocol to behave correctly.

\section{Objectives}
Design and implement a distributed protocol where real world observations can
  be used as blockchain transaction inputs.

\subsection{Specific Objectives}
\begin{enumerate}
	\item Provide a protocol to make possible to gamble with untrusted peers over
	  real world events.
	\item Provide the correct economic incentives to the protocol participants to
	  behave correctly, so everyone incentives are aligned.
	\item Implement a proof of concept of the designed protocol.
	\item Debate of implications and other applications for the designed protocol.
\end{enumerate}

\section{Methodology}
The main phases of this work will be the following:
\begin{enumerate}
	\item Extensive review of existing proposal and implementations to solve
	  the proposed problem or similar ones. As cryptocurrencies are a recent
	  investigation field, this review must cover literature as well as
	  community gathering places, such as forums and specialized blogs, magazines,
	  etc..
  \item Analysis of current solutions to the problem and similar ones.
  \item Design and implementation of a protocol to solve the problem.
    Implementation is considered very important as the current rate of change
    of cryptocurrencies is considerably fast, validating the protocol within a
    real implementation is critical.
  \item Analysis of the economic incentives of the protocol participants, to
    ensure protocol viability.
\end{enumerate}

\section{The Protocol}

The main idea behind this work is to eliminate the more single points of
  trust we can when performing bets. Traditional currencies are produced and
  controlled at Government's will, so the first decision was to use a currency
  without a single controller, we chose Bitcoin mainly for two reasons.
It is the first and one of the most stable currencies out there, changes are
  made much slower than other currencies, the market back this claim by making
  bitcoin the Cryptocurrency with by far the biggest market capitalization.
And second, the network supporting bitcoin is much bigger than the ones for
  other cryptocurrencies. This makes much harder to attack and take control
  of the currency.

There are two mains phases in the protocol, where the first one is optional and
  can be replaced at players will:
\subsection{Oracle's selection}
Bitcoin (like most of the cryptocurrencies) includes a scripting language able
 to control money transferences, well defined and with its execution enforced by
 the complete bitcoin network.
The challenge is to bring data from outside the bitcoin data and reason about
  it. Our protocol relies on several paid ``oracles'' to bring this data.
As the oracles' output will be used to decide who is the bet's winner, it is a
  crucial step to avoid a player getting itself or compromised oracles to decide
  the bet winner.
We say this phase is optional as it might be the case both players trust already
  in a set of oracles.

\begin{enumerate}
  \item The first step is to compile a list of available oracles, we use as
      decentralized database for this list the blockchain. Everyone willing to
      be an oracle can send a transaction to register into the blockchain.
  \item The players negotiate some parameters, as the number of oracles to use
      and the threshold to decide the winner.
  \item In order to decide which oracles to use, the oracles need to pick a
      subset of the available oracles, they do this by running a distributed
      coin tossing protocol.
    With this, they can be sure the compiled list is a random subset of the
      full list. If the list is big enough, the chance of one user controlling
      the oracles gets smaller. As it would be too expensive to control almost
      all the oracles in the list.
\end{enumerate}

\subsection{Bet resolution}
This is phase starts after both players agree the bet with and the oracles to
  be used on it.

\begin{enumerate}
  \item The players send a transaction to the blockchain with the bet
      description, including the IDs of the oracles they want to decide the
      winner.
    We call this transaction ``Bet promise'',as the players commit to the bet by
      placing it. The wage is also on it.
    The other purpose of this transaction is to invite the oracles to
      participate in the bet, we make its ID public so they can identify itself
      and inscribe to participate as oracles.
  \item The oracles will see the transaction inviting them to participate in the
      bet, they will evaluate it and, if they are interested.
    They will reply with a transaction containing a reference to the ``Bet
      promise'' transacation and a small deposit as commitment that they will
      participate in the process.
  \item When the players see the answer from the expected number of oracles,
  	  they will send the ``Bet'' transaction with funds of the bet and the
  	  oracles' reward.
  	If not enough oracles reply to the call, a second invitation can be send to
  	  a different set of oracles to fill the available spots.
  \item As soon as the bet event takes place, oracles are able to collect its
  	  payment from the Bet transaction. This payment gets available by making
  	  public, -voting- by the winner.
  	After the threshold number of oracles collect its payment, the winner player
  	  is able to collect its prize, its private key and the oracle votes are
  	  required to get it.
  \item After a second timeout, players can take the deposit from the oracles
  	  that did not participate in the bet resolution.
\end{enumerate}

The payment of the oracle's is not the only cost of this protocol.
There is a not insignificant number of transactions in the protocol, as there
  is a fee by each transaction in the Blockchain, this makes the protocol more
  expensive.
This is a problem for small bets, the presented protocols is prohibitively
  expensive for bet of just a few dollars. At least with the current fee costs
  of bitcoin.

If we step out a little bit, the proposed protocol uses paid oracles to get a
  binary answer about an event outside the blockchain. This is not useful only
  for betting on that outcome. The oracles are \footnote{\colorbox{red}{
  is this the word?}} insensitive to the use given to their answer, they get
  paid anyway. Further applications of the protocol can be generalized from our
  proposal. \colorbox{red}{Como por ejemplo.... ?}
