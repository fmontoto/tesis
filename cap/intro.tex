\begin{intro}
\section{Gambling}
Gambling is the activity of predicting events and placing a wager on the
  uncertain outcome of those events, with the intent of winning money or
  valuable goods.
A wager can be put on many different events, in a casino we find randomizing
  devices as dices, roulette wheels, etc. which are used to get randomize
  events. In other establishments we can bet on sport events, such as a horse
  racing, football games, etc. or the minimum temperature in Santiago during
  this night.
  
Its popularity and the big amounts of money at stake,
  inevitably entails a lot of interest on this activities. Most of the time
  gambling is heavily regulated and taxed, also it is usual that lotteries are
  owned by the state.

Gambling not only takes place in casinos, lotteries or betting sites, it can
  also involve two or more individuals with no intermediaries. In Chile
  friends usually bet on their favorites football teams.
  
But all the mentioned ways share a common obstacle, participants are required
  to trust in the other parties to pay if they lose.
  Even if the bet takes place in a casino, where the law can enforce the bet,
  is not certain the casino will be able to pay after the resolution.
  
We might not be aware of the fact, but every time we place a bet we are
  implicitly trusting in a third party, either the other player or the bet
  site. For casinos this is usually not a problem, as they are regulated by
  law, any misconduct can get the casino to the justice and even get its
  license revoked.
  Usually this does not represent a problem, 
  
  
\section{Cryptocurrency}
Digital currency refers to any currency stored and transferred
  electronically.
  A subset of the digital currencies is called virtual currencies: them are
  usually defined\citation{bcentraleuro} as a \itquote{ unregulated,
  digital money, which is issued and usually controlled by its developers, and
  used and accepted among the members of a specific virtual community}.
  
Based on the interaction of the currency with currencies outside the
  community there are three types of virtual currencies: The ones with almost
  no interaction with the outside money, this is usually the case of video
  games, where its currency is only valuable within the game. A second
  type is where the currency can be purchased directly using other currency.
  Here, we observe an unidirectional flow. The third type is when the flow is
  bi directional, the users can sell and buy the currency. 
  
A cryptocurrency is a bi directional virtual currency, that uses cryptography
  for security and anti-counterfeiting measures. Virtual currencies are been
  historically linked to cryptography, the first known investigations
  \citation{chaum1983blind} to establish a virtual currency where lead by David
  Chaum, an American cryptographer. However, despite his and others effort
  (e-gold\footnote{https://www.wired.com/2009/06/e-gold/},
   Ecash\citation{chaum1990untraceable},
   DigiCash, LibertyReserve, among others), virtual currencies never where
   massively adopted.
   
By late 2008, using a pseudonym, was released a short
  whitepaper\citation{nakamoto2008bitcoin} with yet another virtual currency
  protocol specification. A few months later, during 2009 its implementation
  was made available as open source code. The main difference with previous
  implementations was its lack of a central organization, this new coin was
  completely decentralized. The software started to being run by some early
  enthusiasts and Bitcoin gave the step from an idea to an usable coin. The
  first years was the coins were exchanged for free among the community users.
  However, at some point the community was big enough and its members started
  to give value to the coin, then the first exchanges from and to other coins
  started to take place. Bitcoin transitioned into a bi directional flow
  virtual coin.

Then the first online exchanges between bitcoin and other currencies started
  to appear, the coin started to gain traction as people outside the community
  were able to buy and sell coins. As the money became popular, the idea was
  taken and a whole generation of cryptocurrencies were born. Today the
  market capitalization of Bitcoin (this is, the amount of money times its
  value in USD) is over 25,000,000,000 USD.
  
\section{Gambling in bitcoin}
Once bitcoin got some traction, several web sites started to offer some
  games of chance an act as online casinos. Where the only different with a
  traditional online casino was the currency on which the bet takes place.
However, as any physical casino, any player who decided to play here is at
  the mercy of the casino. If the casino does not want or does not have the
  means to pay, there is nothing the participant can do and its money is lose.
 
After online casinos came the betting platforms, were you could also do sport
  or other real world events bets. This introduces a new uncertainty in the
  process: Who gets to decide the outcome of the event? The answer is the same
  all the time, the same casino observes the outcome of the bet and decides
  who wins.
  
  
  

 
 




\end{intro}
