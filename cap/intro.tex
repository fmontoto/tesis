\begin{intro}


\section{Electronic Money}
Electronic mone

\section{Gambling}
Gambling is the activity of predicting events amd placing a wager on the
  uncertain outcome, with the intent of winning money or valuable goods.
  Among the most common gambling forms we can find lotteries, casino and sport
  results.

Usually there is a big amoung of money or valuable goods at stake, this
  inevitably intails a lot of interest on them, and governments are not the
  exception.
  Gambling is heavily regulated and taxed it is also not unusual that
  lotteries are owned by the state. Legal gambling can provides significant
  government revenue in some states, as Monaco or Macau, China.

Gambling not only takes place in casinos or lotteries, it can involve two
  or more individuals with no intermediaries.
  But both ways share a common obstacle, they need to trust in the other
  parties to pay if they lose.
  Even if the bet takes place in a casino, where the law can enforce the bet,
  is not certain the casino will be able to pay after the resolution.




Una apuesta es una forma de juego basada en el azar o la predicci\'on de eventos
  futuros, en la que toman parte al menos dos participantes. Cada participante
  realiza una predicci\'on, disjunta de la del otro participante, sobre el evento y
  quien acierta es el ganador. El ganador recibe un beneficio pactado al momento de
  realizar la apuesta, en desmedro del perdedor. Usualmente el beneficio es dinero
  o un bien valioso, que el otro jugador provee.

Un problema inherente a las apuestas es la resoluci\'on de \'esta, es decir que el
  perdedor transfiera lo apostado al ganador. Una soluci\'on para este problema
  es transferir los bienes apostados a una tercera entidad en la que ambas
  partes confíen, para que una vez resuelta la apuesta haga llegar los bienes al
  ganador. Otra soluci\'on es apostar en entidades reguladas, que de negarse
  a pagar tras perder el ganador puede concurrir a las autoridades encargadas
  de hacer cumplir la ley para obligar al perdedor a pagar. Las entidades m\'as
  conocidas bajo esta modalidad son las casas de apuestas y los casinos.

Las tres formas de apuestas descritas comparten una debilidad com\'un, todas
  requieren que los apostadores confíen en otro ente. En el primer caso y m\'as
  simple, cada participante debe confiar en el otro. En el segundo, todos los
  participantes confían en un tercero, que no participa de la apuesta, para la
  resoluci\'on de \'esta. En el tercer caso la confianza recae en el ente regulado
  y en las instituciones encargadas de aplicar la ley, que poco podr\'an hacer en
  caso de que el ente regulado haya gastado o perdido el bien.




Usualmente cuando hablamos de dinero imaginamos un billete o una moneda. Sin
  embargo, cuando decimos dinero digital es m\'as complejo hacernos una imagen.
  Generalmente lo asociamos a pagos electr\'onicos, en los cuales no vemos el
  billete o moneda. El dinero digital exhibe propiedades similares al f\'isico,
  se diferencia en que permite transacciones instant\'aneas e intercambios
  entre distintos pa\'ises sin importar fronteras.

Una criptomoneda es un tipo de moneda digital, donde se utilizan medios
  criptogr\'aficos para asegurar las transacciones y para controlar la
  creaci\'on de nuevas unidades. La idea de utilizar herramientas
  criptogr\'aficas en monedas digitales surge como t\'opico de
  investigaci\'on en los a\~nos 80, cuando David Chaum \cite{chaum1983blind}
  introduce una nueva primitiva para realizar pagos imposibles de rastrear.

En el a\~no 2012 el banco central europeo \cite{bcentraleuro} define una moneda
  virtual como: \enquote{un tipo de moneda digital desregulada, que es emitida y
  usualmente controlada por sus desarrolladores, usada y aceptada entre los
  miembros de una comunidad virtual espec\'ifica}. Monedas virtuales
  seg\'un esta definici\'on existen desde hace tiempo, principalmente en
  comunidades de juegos.

Hasta el a\~no 2008 todas las monedas digitales conocidas, tanto en
  circulaci\'on como ya retiradas, compart\'ian una cualidad fundamental con
  el dinero f\'isico. Eran completamente controladas por una entidad central,
  tal como en el dinero f\'isico lo hace el banco central. Durante el mes de
  Noviembre del 2008 Satoshi Nakamoto publica ``\textit{A peer-to-peer
  electronic cash system}'' \cite{nakamoto2008bitcoin}, la primera moneda
  digital completamente descentralizada\cite{brito2013bitcoin}:
  \textit{Bitcoin}.

Tan solo unos meses despu\'es de su publicaci\'on ser\'ia puesto a
  disposici\'on de la comunidad un software implementando el protocolo
  propuesto. Empezaba as\'i a circular una moneda que cambiar\'ia radicalmente
  el escenario de las monedas digitales.


\end{intro}
