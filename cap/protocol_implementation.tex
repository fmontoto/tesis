\section{Implementation}

Bitcoin is widely used as value storage, most of the applications and library
  supporting it provide just a wallet and blockchain inspection functionality.
Creating custom transactions is not a common functionality in bitcoin clients,
  and by the time this work started there was no library available to generate
  custom transactions.

In order to prove the viability of the proposed protocol and generate the
  transactions an implementation
  was written (see Appendix~\ref{sec:appendix_transactions}).

We used Java to implement all the protocol functionality.
We divide our implementation in three logical modules to explain the work done:
\begin{enumerate}
\item \textbf{Bitcoin communication:}
The official Bitcoin client exposes an RPC server to interact with the
  blockchain.
This module provides an interface to the Bitcoin client, it allows our
  implementation to get information from the Blockchain and verify the
  validity of the transactions we generate.

\item \textbf{Bitcoin core:}
This module exposes Bitcoin objects with our required functionality.
It is capable to generate custom transaction and sign them, it also
  provides transaction parsers and human readable views of them.
Likely all this functionality would be in a Bitcoin library if bitcoin
  were usually used to do smart contracts.

\item \textbf{Protocol implementation:}
In this module we included the protocol specific functionality, it is able
  to generate and understand the Bets and the protocol transactions.
It also exposes a distributed coin tossing implementation to select oracles
  randomly from a list.
\end{enumerate}

Our implementation has a little bit more than 10 thousand lines of code.
It includes:
\begin{itemize}
\item An encrypted and authenticated communication channel, which can start from
  an insecure channel only knowing the Bitcoin address of the other
  participant.

\item A blockchain scan to compile a list of oracles from its Registration
  transactions.

\item A distributed coin tossing implementation to select the oracles to use
  from the compiled list.

\item A direct method to generate each of the required transactions in the
  protocol, including all the redeem transactions required to spend them.

\item A check to verify the transactions received from the other player are
  valid and include the expected inputs and outputs.
\end{itemize}

We used this implementation and the official bitcoin client to prove the
  validity of the transactions generated.
We want to be sure the transactions generated are valid Bitcoin transactions,
  so testing them using the official client is a crucial step.
We created bitcoin accounts for all the participants and simulated a
  protocol run by generating all of its transactions, using each participant's
  accounts.
After the transactions are generated and signed with all the required keys,
  they are sent to the official bitcoin client for verification.
This is the most important step, as this verification is exactly the same one
  used to verify transactions in the real blockchain.

Due to lack of libraries doing custom transactions, this implementation is
  a contribution not only for this protocol, but potentially for other
  applications using custom transactions.

The implementation~\cite{implementation} is available at
  \url{https://github.com/fmontoto/thesis_oracle} under the MIT license.
