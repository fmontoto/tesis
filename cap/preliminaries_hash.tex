\section{Hash}
Hash is an overload word and it is usually used to define various things:

\subsection{Hash Function}
It is a function able to map data from arbitraty size to data of a fixed size.
The range has only elements of a fixed size, so it is bounded by all the
  elements of that size.
If we represent the data in a binary base, the range is bounded by $2^n$ where n
  is the size in bits of the output.
The domain of the function is unbounded, by the \textit{Pigeonhole
  Principle:}
\begin{equation}
	\exists \, i, j \mid f(i) = f(j), \, i \neq j
\end{equation}
When this happen, we call it collision, and for most uses of a Hash function are
  unwanted.

Hash functions are used for many things: File comparison, instead of comparing
  files bit to bit, the image of a Hash Function can be compared instead;
  Hash-Tables, this allows quick lookups for the elements; Finding similar
  records, by using a Hash Function that produces similar images for similar
  pre-images, etc..

\subsection{Image of a Hash Function}

If not stated otherwise we will use the word ``Hash'' to denote the image of
  some data using a Hash Function.

\subsection{Cryptographic Hash Function}
This refers to a special class of Hash Functions the cryptography has defined
  to be suitable for its use on cryptographic applications. These functions are
  designed to be ``one way'' functions, meaning it is unfeasible\footnote{We %
  say something is computationally unfeasible when even it is computable, it
  will require far too many resources to do it.} to invert if the input to the
  function is chosen uniformly at random.

In an ideal cryptographic function, the most efficient way to find one of the
  preimages is by a brute-force search\footnote{Also known as exhaustive search,
  consisting of enumerating all the potential solutions and checking which of
  them satisfies the predicate}. We call this property
  \textit{preimage-resistance}.
It is also important for this ideal function to be \textit{collision
  resistat}, this means it is unfeasible to find any two distinct inputs
  $x, x'$ with the same image, i.e., such that $h(x) = h(x')$.

When using this ideal function, on average, producing a (second) preimage
  requires $2^n$ operations, and producing a collision requires at least
  $2^{n / 2} $ operations \cite{preneel1993analysis}.
