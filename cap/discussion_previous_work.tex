\section{Comparison with Existing Solutions} \label{sec:discussion_previous_work}

This protocol is not the first attempt to perform trustless gambling using
  crypto currencies.
Is not even the first one to do it using Bitcoin in particular.
In this section, we try to differentiate this proposal from the existent
  solutions by highlighting their differences.

In Section~\ref{sec:previous_work} we presented a short introduction to the
  existent solutions.
Oraclize and Town Crier provide a secure interface to web sites.
Making information of these websites available in the blockchain.
A web site is a centralized data source, however combining multiple sites as
  data sources and considering a majority instead one of them give us a
  decentralized source.
This could be a really good approach to do bets on popular events, as
  many websites already publish results for this events (football results,
  presidential election results, etc..).
If the outcome is used from a massive number of bets, the cost of the oracles
  could be amortized among all the players.

There are two main problems with this approach: none of the discussed
  website interfaces provide a strong cryptographic proof binding the results
  and the website; and is difficult for this approach to work with non popular
  events, as web sites does not provide results for them.
An important difference with the proposed protocol is the platform they work
  on, Ethereum.
In the next paragraphs we discuss how this might be an important difference.

Winsome is another implementation on top of Ethereum that allows people to
  gamble against each other, but on random events.
It emulates casino games as blackjack and roulette.
As its target are random games, it's not suitable for real world event bets.

Ethereum unlike Bitcoin was not designed with the goal of being a currency,
  but a platform to write and enforce code execution in a distributed
  way.
It's primary objective is to become a \textit{blockchain app platform} as they
  state in their website\footnote{https://ethereum.org/}.
Because of this, their scripting language is Turing complete and provides much
  more flexibility.
This makes Ethereum the home for most of the blockchain applications.
Recently, one of the Ethereum biggest app, the Decentralized Autonomous
  Organization (DAO), was subjected to an attack were attackers gained control
  of one third of the money it had.
About 50 USD Million at the time of the attack.
Most of the Ethereum community decided to ``rollback'' the history and give the
  DAO its money back.
Immutability of the coin was therefore, broken.

On the contrary, bitcoin does not aim to be an app platform.
People try to overcome its scripting limitations and write applications for it.
Orisi is one of this applications that aims to solve the same problem we do.
It works on top of bitcoin, one key difference is it's centralized oracle
  database.
Other important difference is its secrecy, it uses BitMessage to keep the
  location and identity of the participants secret.
Although we could use BitMessage, our implementation uses direct TCP
  communication among the participants.

Probably the biggest difference between our proposal and Orisi are the type of
  transactions used by each one.
Orisi uses multisignature address to move the bet money into the control of
  the oracles and the players.
This means, in order to redeem the prize, the winner needs to make a transaction
  and get it signed by a majority of the oracles.
So oracles need to agree in the destination of the prize, otherwise they will
  not sign the transaction.
Our proposal gets the oracles out from this decision and requires only the
  winner signature to spend the money.
In order to claim its payment, oracles are required to reveal one of the two
  secrets they kept.
When a majority of the oracles have revealed the secrets corresponding to the
  winner, players and oracles collect them all and claim its prize.

This has two main advantages: the first one, already mentioned, keeps the
  oracles out from deciding where does the prize go.
Oracles are not involved in the transactions spending the prize;
The second one is how it bounds the action of emitting the vote for the winner
  with getting paid.
There is no pressure for the oracle in responding before the other do it,
  there is a fixed timeout defined from the beginning to vote.
Oracles can be sure if they behave properly and vote on time they will be
  remunerated, even if they vote after a majority already did.
In the multisignature address, the player could decide to wait only to the
  minimum required number of oracles and send the transaction to redeem the
  prize.
Then the prize gets splitted in exactly the required majority.
Oracles that answer on time after this are not guaranteed to get paid.
