\section{Digital Signatures}
The idea of ``Digital Signature'' was introduced in 1976 by Diffie and
  Hellman in ``New Directions in Cryptography''\cite{diffie1976new}. In this
  work is also introduced what they called ``Public Key Cryptosystem'', were
  enciphering and deciphering operations use different keys, $E$ and $D$, such
  that computing D from E is computationally infeasible. Today this pair are
  widely used and are known as Public Key (PK) and Secret Key (SK).

The public key cryptosystem, or asymmetrical cryptography was created to solve
  one important problem of symmetrical systems\footnote{As opposed to the
  asymmetrical one, this system uses the same key to cipher and decipher
  the messages.}: It is imposible to start a secured communication in an
  insecure channel without previously exhange of a key using a secure channel.
To stablish a secure communication within an insecure channel participants
  makes its PK publicly available to the others. Anyone willing to talk to
  another participant must cipher its message using the public key of the
  receiver, this way the only one able to decipher the message is the intended
  receiver.

A digital signature, as its name indicates, is a mechanism to provide protection
  against third paty forgeries. It must be easy to for anyone to recognize as
  authentic but impossible for anyone but the signer to produce it. This is
  specially challenging since any digital signal can be easily copied.

It works within the public key cryptosystem the signer uses its SK to produce
  a signature over the message to sign, and anyone with the signer PK and the
  message can determine the validity of the signature.
