\section{Ecash}
Digital currencies have been a research topic since at least 1983 when David
  Chaum \cite{chaum1983blind} introduced Blind Signatures. A form of digital
  signature where the content of the signed message is blinded, so the entity
  signing the message do not get to see it. This technique was used to provide
  untraceable payments in a cash system where however anybody can check the
  signature is valid.

The field has been an active topic in the academy and as business intent.
  Much research has been published proposing new schemes and
  cryptographic primitives \cite{okamoto1991universal}\cite{chaum1992achieving}
  \cite{boly1994esprit}\cite{anderson1996netcard}\cite{lysyanskaya1998group}.

In 1990 David Chaum founded DigiCash, which developed an early electronic
  payment based on blind signatures. Payments using the software were
  untraceable by the issuing bank or any third party, including the government.
However, the company was not able to beat credit cards in the electronic
  commerce and files its bankruptcy in 1998.

In 1996 e-gold allowed its user to buy electronic money (``grams of
  gold''\footnote{Also of platinum, silver, etc..} that were backed by precious
  metals held by the company \cite{hughes2007developments}.
The users can buy, sell and transfer the ownership of the metals over the
  Internet. In 1999 the \textit{Financial Times} described e-gold as the only
  electronic currency that has achieved critical mass on the web.
However, its success contributed to its demise. It was used for fraud, phishing,
  cyber crime gangs, etc.. Law enforcement agencies began to characterize
  e-gold as the favorite payment system for criminals and terrorists\footnote{%
  "Feds out to bust up 24-karat Web worry". NY Daily News. 2007-06-03. Retrieved
  2017-07-27}. By 2007 the justice started to seize e-gold balances that ended up
  with the suspension of the service.

Other initiatives suffered the same fate.
Closed by the governments or for lack of interest are the reason for the
  failure.
New cryptocurrencies are virtually impossible to close by a government because
  its distributed design, and so far people have been very enthusiastic about
  them.
This might signify they will prevail much longer than previous solutions.
