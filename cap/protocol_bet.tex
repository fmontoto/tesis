\subsection{Second part: The bet}

\subsubsection{Bet Promise}
Once players decided the oracles to use, whether using the \textit{Oracle
  Selection} part or by any other mean.
They need to agree in the terms of the bet, the event and who is the winner on
  each outcome, the time available for the oracles to answer, money required by
  each player, fees of the  oracles, deposit required to the oracles, etc..
Once all the bet parameters are set, they are serialized and its hash is
  calculated, players puts all the money required to run the protocol,
  including fees and the prize into a transaction.
The selected oracles, bet hash and a method\footnote{For instance an URL to a
  website with the bet description. Oracles are responsible to check the
  description fetched with the hash provided in the transaction.} to get the
  transaction full description are appended to this transaction in plain text
  and the transaction is sent to the blockchain.
We call this transaction ``\textit{Bet Promise}'', as it is a commitment from both
  players to the bet.

\colorbox{red}{Figura de la tx}

The commitment consists in paying the transaction fees\footnote{This money is
  gone forever.}, and to move all the money required by the BET into the control
  of both players.
Any other transaction with this money needs the approval from both, however in
  the case one of the player dissapear and does not participate anymore in the
  protocol, most of it is return \colorbox{red}{Hablar de la plata que va a los
  oraculos y es irrecuperable, y el costo de hacerla recuperable}.

\subsubsection{Oracle Inscription}
Oracles invited to participate needs to retrieve the Bet description as
  instructed in the \textit{Bet Promise} transaction and decide whether to
  participate or not.
When an oracle does not participate, players decide how to select a new one,
  a waiting list is recommended to be selected in the first part of the
  protocol.
When it decides to participate, it builds and send to the players
  \colorbox{red}{Explicar como se comunica} the ``Oracle Inscription''
  transaction.


This transaction is the oracle's commitment with the bet, and its acceptance
  from the players. The inputs come from the Oracle's deposit and the players
  joint account, some of it goes to an account controlled by the oracle and
  the players.
The portion left goes to any of the players if the oracle does not give the
  correct answer or goes back to the oracle after the bet is done, we call
  this output ``two answers penalty'', as it is a penalty for the oracle
  when it misbehaves.

This transaction also binds the oracle anwers, at this step the oracle generates
  two random strings to be used as answers, one for each possible outcome.
This strings are to remaing secret until the oracle answers the event's
  outcome.
However the hashs of this strings are included in the \textit{two answers
  penalty} output, if a player gets to know both answers can spend this output.

\subsubsection{Bet}

Once the required oracles are inscribed to participate the ``Bet'' transaction
  is built by the players.
 The input contains what is remaining from the original players contributions to
   the prize, controlled by the players' joint account plus most\footnote{The
  other portion of this deposit was used in the \textit{two answers penalty}
  output in the \textit{Oracle Inscription} transaction.} of the oracles
  deposit, controlled by the corresponding oracle and both players.
Because of this, the transaction must be signed by both players and all the
  oracles.

The first two outputs are the prize, divided in two, so in case there is no
  answer from the oracles half of the prize returns to each player.
Each of them getting back almost all the money initially spend, some is lose
  on fees and oracle payments. \colorbox{red}{Esto depende de los valores de
  las variables}. On the other hand, if the oracles get to an agreement and the
  bet is resolved on time. Using the oracles' answer the winner player can spend
  both outputs.

The next $n$ outputs are the oracle payments, they are spendable by the
  corresponding oracle plus any of its answers.
Spending this output requires to make public the oracle's answer. This way the
  oracle is guaranteed to get its payment when responding on time, and players
  know the oracle only gets pay when it answers.
  If the transaction is not spent in a defined time (T$_{\text{reply}}$) we say
  the oracle didn't answer and the money can be spend by the players joint
  account.
This prevents the oracle of taking the money after not responding in the
  required time.

The last $n$ outputs are a withholding for the same amount of the oracle
  payment, if the oracle gives a wrong answer for the oucome this money
  goes to the real winner, this way we take the payment out from the oracle.
If it behaves as expected, the oracle can spend this money some time after
  (T$_{\text{undue}}$) the bet is resolved.

After this transaction is placed, every player only wait for the events to take
  place and spend, if they can the outputs availables. The only transaction
  remaining are the participants taking the money they are entitled to.



The payment of the oracle's is not the only cost of this protocol.
There is a not insignificant number of transactions in the protocol, as there
  is a fee by each transaction in the Blockchain, this makes the protocol more
  expensive.
This is a problem for small bets, the presented protocols is prohibitively
  expensive for bet of just a few dollars. At least with the current fee costs
  of bitcoin.

If we step out a little bit, the proposed protocol uses paid oracles to get a
  binary answer about an event outside the blockchain. This is not useful only
  for betting on that outcome. The oracles are \footnote{\colorbox{red}{
  is this the word?}} insensitive to the use given to their answer, they get
  paid anyway. Further applications of the protocol can be generalized from our
  proposal.
Resolution of contractual disputes is an interesting topic, where parties agree
  to use arbitrator(s) to decide a missunderstanding.
In this case oracles takes part in the protocol more like judge than a oracle.
